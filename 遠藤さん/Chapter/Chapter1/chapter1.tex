\chapter{序論}
高エネルギー加速器実験では、加速器を用いて粒子にエネルギーを与え、加速粒子同士の衝突から高エネルギー状態を作り上げる.その高エネルギー状態から生成される終状態粒子の情報を崩壊過程から研究し、素粒子間の相互作用の物理を探索する.

崩壊過程を探るには粒子の崩壊位置を高精度に測定することが重要であり、崩壊点検出器を衝突点の最近傍に配置する必要がある.そのため、検出器にはコンパクトさ、高速性、高い位置分解能が要求されることから、エネルギー分解能に優れ、小型で比較的速いタイミング特性を示す半導体検出器を用ることが多い.しかし近年の高エネルギー実験では希少崩壊の観測精度を上げるためにルミノシティを上げて統計数を稼ぐ必要があり、このような実験の高度化に伴い既存の半導体検出器ではなく新たな検出器開発が必須になった[1].

こうした要求に対応すべく、2005年よりKEK測定開発室のSOIPIXグループによるSilicon On Insulator(SOI)技術を用いた次世代を担う半導体検出器の開発が行われており、筑波大学はプロジェクト立ち上げ当初から共同研究を進めてきた.SOI技術を用いたSOIピクセル検出器は低物質量・低消費電力に加え高放射線耐性など多くの利点があるが、本研究ではピクセルの細密化によってとりわけ優れた位置分解能が見込まれるFPIX2の評価を行った.本論文の流れを以下に示す.

2章では半導体の基本特性や検出器として使用する際の特性を紹介し、3章ではSOI検出器の詳細を述べる.研究内容は2章から構成されており、まず4章でFPIX2とは支持基板の異なるINTPIX2との収集電荷率の比較を行う.その後、5章でFPIX2の放射線耐性と位置分解能についてテストビームによる評価を述べる.最後に研究の要点・今後の展望を6章でまとめて本論分を締める.
