\begin{abstract}
素粒子物理学においてニュートリノは1998年のニュートリノ振動の発見以来、大きく発展している分野である。
ニュートリノ振動の観測により、ニュートリノに質量があることが示され、3種類のニュートリノの質量自乗差とニュートリノ混合角が高精度で測定されている。
しかし、各種類のニュートリノの質量自身は測定されていない。
\par
重いニュートリノは軽いニュートリノに光子を伴い崩壊する。
これをニュートリノ崩壊といいい、崩壊光を測定することにより、重いニュートリノの質量を決定することができる。
ニュートリノの寿命は標準模型においては$~10^{43}$年と非常に長いため、ニュートリノ崩壊は非常に稀な現象である。
また、予想されるニュートリノ崩壊光のエネルギーは$m_3=50[meV]$と仮定すると$E_\gamma=25meV $となる。
これは波長にすると$50\mu m$ほどになり、地上では黄道光に埋もれてしまう。
\par
我々COBAND$(\bm{Co}smic\ \bm{Ba}ckground\ \bm{N}eutrino\ \bm{D}ecay\ Search)$グループでは、宇宙背景ニュートリノの崩壊光を測定し、ニュートリノの質量を求める。
宇宙背景ニュートリノの探索を行うロケット実験では遠赤外線1光子ごとにエネルギーを2$\%$以下の精度で測定する必要がある。
しかし、本実験に用いる予定のNb/Al-STJ 検出器は$E_\gamma=25meV\ $に対して10$\%$ほどの分解能しかないため、1光子のカウントは可能だが、入射した光子のエネルギーを測定することは困難である。
そこでロケット実験ではニュートリノ崩壊光をブレーズド回折格子によって波長ごとに分光しNb/Al-STJ検出器に入射することで崩壊光を測定する。
\par
現在、ロケット実験に向けた回折光子の設計が進行しているが、設計に利用するシミュレーターの妥当性を検討する必要がある。
ここでは、福井大学にある遠赤外線レーザーを回折格子に入射した結果から、シミュレーターの妥当性を検討した。
また、ロケット実験に向けた回折格子の設計を行った。
\end{abstract}
\newpage
