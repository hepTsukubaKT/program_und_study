\begin{abstract}

2005年より高エネルギー加速器研究機構(以下KEK)を始めとする共同研究グループでSOI(Silicon On Insulator)技術を用いた次世代型ピクセル検出器の開発が行われている。SOI検出器は回路部と支持基板部のシリコン層間に${\rm SiO 2}$の絶縁層を埋め込むことにより、各々のトランジスタを分離することが可能である。
この特徴から、(1)モノリシック型検出器、(2)SOI CMOSによる読み出し処理回路、(3)高比抵抗シリコンによる検出部の全空乏化が可能といった大きな利点をもち、中でも従来のピクセル検出器に比べピクセルの細密化による高い位置分解能などが見込まれる。

多くのメリットをもつSOIピクセル検出器だが、高エネルギー実験分野での実用化には放射線の蓄積ダメージであるTotal Ionizing Dose(以下TID)効果による回路部の特性変動が長らく課題とされてきた。しかしTID効果は放射線に曝されることで酸化膜中に電荷が蓄積されることが原因のため、酸化膜中にシリコン層を埋め込む2層埋込酸化膜構造を導入することで膜中シリコン層に外部から電位を与え変動の補償を行うことを可能にした。

本研究は、高い放射線耐性をもつSOI検出器のうちピクセル回路部の電荷収集を担っていたストレージキャパシタを取り除くことで、より一層のピクセル細密化を実現したFPIX(Fine PIXel detector)2についての評価を行った。まず同様のSOI技術を用いた電荷積分型のピクセル検出器のうち旧型であるINTPIX(INTegral PIXel detector)h2との電荷収集効率の比較について報告する。また、2017年1月23日から2月7日にかけてアメリカ・フェルミ国立加速器研究所で行われたビームテストの結果からFPIX2の位置分解能依存性について述べる。


\end{abstract}
\newpage
