\chapter*{謝辞}
本論文の完成に至るまで、多くの方々からご指導いただきました。ご協力いただいた皆様に心から感謝申し上げます。

本研究を進めるにあたり、私が学類4年生の頃から今まで3年間、金信弘教授と武内勇司准教授からご指導いただきました。
大変感謝しております。
金先生からは、私が研究して行き詰まったときに親身になって相談に乗っていただきました。
またCERNの宇宙史拠点実習や学会$\cdot$研究会など貴重な経験を数多くさせていただきました。
武内先生にはミーティングの際にテクニカルなアドバイスを数多くいただきました。
分からないことだらけの私に対して丁寧にアドバイスしていただいたのにも関わらず、頓珍漢なことをしてしまい実験に失敗してしまったこと多々ありました。
申し訳ありませんでした。

指導教員の原和彦准教授には特に本論文の際、丁寧で的確なアドバイスをいただきました。ありがとうございました。

佐藤構二講師と大川英希助教には宇宙史拠点実習ではお世話になりました。
特に佐藤先生にはあまりrootに触れてこなかった私にrootの基本操作やcharged higgsの解析について親身になって教えていただきました。

KEK\ SOIグループの倉知郁生先生には、極低温環境用SPICEパラメータ抽出のためのMOSFETのIV測定の際に本当にお世話になりました。
IV測定がうまくいかず行き詰まったとき、わざわざ筑波大まで来ていただきご指導いただきました。
JAXA\ 馬場俊祐様には、我々が測定した電流電圧特性をもとに実際にパラメータ抽出していただきました。
お二人なしでは本論文を書くことはできませんでした。
本当にありがとうございました。

筑波大STJグループの方々には、主に日々の実験の際に本当にお世話になりました。
武政さんには装置の仕組みや物理についてなど私の身になる様々なことを教えていただきました。
また昼に美味しい店に連れて行っていただけたのもいい思い出です。
昨年度まで在籍していた理研の木内さんには私が学類4年のころから、日頃実験に失敗したり頓珍漢なことをして迷惑ばかりかけてしまいました。
しかし、木内さんの愛のこもった厳しいご指導のおかげで、かなり根性ついたと思います。
名古屋大の博士課程に進学し研究なされている奥平さん、社会人として働かれている先崎さん、森内さんには、STJだけでなく実験の細々としたことの多くを学びました。
後輩の若狭さんは入学したての頃からSTJグループの即戦力として元気にバリバリ研究頑張ってくれました。
色々と雑用を押し付けてしまい申し訳なかったです。
来年は筑波大STJグループの学生リーダーとして、新たに迎えるメンバーと実験頑張って下さい。
ただ夜遅すぎる研究生活になりがちだと思いますが、体調には十分気をつけて下さいね。

研究室の先輩、同期、後輩にも大変お世話になりました。
社会人として働かれている淵さんとは居室が同じだったということもあり、毎回トイレに一緒に行っていました。
居室からトイレの行き帰りはたった3分ぐらいだと思いますが、他愛のない話で盛り上がっていました。
D3の笠原さんとはいつも一緒にいた気がします。
日々の研究生活だけではなく、人生がかかった就職活動の苦楽も共にしました。
私の喜怒哀楽に常に付き合ってくれた笠原さんは私にとって先輩ですが、親友のような存在です。
来年度からはそれぞれ違う道に進みますが、これからも仲良くしていただけると嬉しいです。

同期の皆とはあまり深く語りあうような間柄ではなかったように思います。
ただ、皆がそれぞれ一生懸命研究しているのを見て私も刺激を受けたし、同期の仲間がいたことで心が救われたこともありました。
皆就職先が決まり、それぞれが違った道を歩むことになりますが、3年間苦楽を共にした皆とはたまには会いたいです。

後輩のみんなとは居室が違かったからか、あまり会うことがありませんでした。
D進するか、それとも就職活動するかどうかわかりませんが、みんな仲良く元気に研究生活を送ってください。

最後になりますが、大学院まで学校に通えたのは何よりも家族のおかげです。
家族には経済的なサポートのみならず、精神的なサポートもしていただきました。
これからは社会人として一からスタートになりますが、人生に悔いがないよう精一杯生きていこうと思います。
24年間、ここまで育てていただき本当にありがとうございました。
\begin{flushright}
八木 俊輔
\end{flushright}
