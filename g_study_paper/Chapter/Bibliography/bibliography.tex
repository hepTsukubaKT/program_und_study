\begin{thebibliography}{20}  % [#] means the num. of bibs
  %\bibitem{キー1} 参考文献の名前・著者1
  	\bibitem{1} 大学共同利用機関法人 高エネルギー加速器研究機構、webpage、『https://www.kek.jp/ja/NewsRoom/Highlights/20120727150000/』\ .
  	
  	\bibitem{2} S.H. Kim, K.Takemasa, Y.Takeuchi, and S.Matsuura,『Search for Radiative Decays of Cosmic Background Neutrino using Cosmic Infrared Background Energy Spectrum』,Journal of the Physical Society of Japan,81 (2012) 024101\ .
	
	\bibitem{3} 阪大物理学オナーセミナー、webpage、『http://osksn2.hep.sci.osaka−u.ac.jp/naga/kogi/handai−honor07/8−nu−property.pdf』\ .
	
	\bibitem{4} 丹羽雅昭、『超伝導の基礎』、東京電機大学(2002)\ .
	
	\bibitem{5} 美馬覚、『ミリ波$\cdot$サブミリ波検出用アンテナ結合伝送型超伝導トンネル接合素子の開発』、博士論文、岡山大学大学院(2013)\ .
	
	\bibitem{13} 浮辺雅宏、『超伝導トンネル接合素子を用いた高分解能X線検出器の研究』、博士論文、東京大学大学院(1998)
	
	\bibitem{6} T.Wada, H.Nagata,H.Ikeda,Y.rai,M.Ohno,K.Nagase,『Development of Low Power Cryogenic Readout Integrated Circuits Using Fully-Depleted-Silicon-on-Insulator CMOS Technology for Far-Infrared Image Sensors』,J.Low.Temp.Phys.,167 602(2012)\ .
	
	\bibitem{7} Behzad Razavi 著  黒田忠広 訳、『アナログCMOS集積回路の設計 基礎編』、丸善株式会社(2000)\ .
	
	\bibitem{8} 先崎蓮、『ニュートリノ崩壊光探索のための超伝導トンネル接合光検出器及び極低温増幅器の開発研究』、修士論文、筑波大学大学院(2016)\ .
	
	\bibitem{9} 森内航也、『ニュートリノ崩壊光探索実験のためのニオブとアルミニウムを用いた超伝導トンネル接合光検出器の性能評価』、修士論文、筑波大学大学院(2016)\ .
	
	\bibitem{10} 奥平琢也、『ニュートリノ崩壊光探索のためのニオブとアルミニウムを用いた超伝導トンネル接合素子光検出器の開発研究』、修士論文、筑波大学大学院(2015)\ .
	
	\bibitem{11} 笠原宏太、『ニュートリノ崩壊からの遠赤外光探索のためのSOI-STJ一体型検出器の開発研究』、修士論文、筑波大学大学院(2014)\ .
	
	\bibitem{12} 武政健一、『ニュートリノ崩壊探索実験のためのハフニウムを用いた超伝導トンネル接合素子検出器の開発研究』、修士論文、筑波大学大学院(2008)\ .
\end{thebibliography}
