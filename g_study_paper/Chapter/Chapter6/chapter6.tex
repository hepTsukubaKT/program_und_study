\chapter{結論}
	ニュートリノは荷電レプトンと弱アイソスピン二重項をなし三世代で構成されている。
	ニュートリノはニュートリノ振動実験による質量を持つことが証明されているが、未だその絶対質量を決定するには至っていない。
	そこで、我々はニュートリノ崩壊で放出される光子のエネルギーを精密に測定することで、ニュートリノの絶対質量を求めることを考える。
	
	ニュートリノの寿命は非常に長いため、観測には大量のニュートリノ源が必要となる。
	そこで宇宙初期に大量に生成されたと予言される宇宙背景ニュートリノを測定に用いる。
	シミュレーションの結果、検出器には25meVの光子に対して2\%以上のエネルギー分解能が要求されていることが判明し、この要求を満たす検出器として我々はNb/Al-超伝導トンネル接合素子光検出器(Nb/Al-STJ)検出器を用いる。将来的には、Nb/Al-STJ検出器と$\mathrm{^{3}He}$減圧冷凍機とその他光学系を搭載したロケットを宇宙空間へ飛ばし、宇宙背景ニュートリノ崩壊探査実験(COBAND実験)を行う予定である。
	
	我々はCOBAND実験に用いるNb/Al-STJ検出器の研究開発を行ってきた。
	産総研CRAVITY製Nb/Al-STJ検出器のリーク電流は、崩壊光1光子検出に必要な100pA以下を達成している。
	しかし、STJ検出器信号が冷凍機外への長い信号読み出し配線系での雑音に埋もれてしまい、遠赤外光1光子検出には至っていない。
	そこで我々は極低温環境下で動作するSOI前置増幅器を導入し、STJ検出器直近で信号増幅し、冷凍機外へ読み出す。
	
	STJ検出器信号増幅のためのSOI前置増幅器への設計要求は、
	\begin{itemize}
		\item Nb/Al-STJ検出器信号帯域(数$\mathrm{\mu s}$〜数百$\mathrm{\mu s}$程度)の高速な信号でも増幅可能である
		\item 冷凍機配線容量(約0.5nF)を駆動可能である
		\item 極低温環境下でも増幅器として動作可能である
		\item 消費電力が$100\mathrm{\mu W}$〜$0.25\mathrm{W}$である
	\end{itemize}
	ことが要求される。
	これら要求を満たすためには入念な回路シミュレーションが必要不可欠であるが、現在極低温環境下のSOI回路シミュレーターが存在しない。
	そこで、我々が前置増幅器に用いるFD-SOI-MOSFETトランジスタのSPICE回路シミュレーターを構築し成功した。
	しかし、
	\begin{itemize}
		\item ドレイン電圧が低い線形領域におけるドレイン抵抗異常
		\item ドレイン電圧が高い領域でのkink効果
	\end{itemize}
	という極低温環境特有の電流電圧特性をシミュレーターで表現することができず、それらの領域を除いた領域でのみシミュレーションは有効である。

	本研究で、LDD不純物濃度を濃くすることで線形領域でのドレイン抵抗異常を抑制することに成功した。
	これにより、より精度の高い極低温環境用SPICE回路シミュレーターを構築できると期待される。
	今後、LDD不純物濃度を改良したSOI-MOSFETで様々なサイズのSOI-MOSFETの電流電圧特性を測定し、サイズ依存性、温度依存性などを検証する予定である。
	
	次に、SOI-STJ4を用いたSTJ信号増幅試験を行った。
	その結果、レーザー照射によるSTJ検出器信号はSOI-STJ4に伝送され、伝送された信号がSOI-STJ4で反転増幅されたことを観測した。
	極低温環境下(300mK)において、SOI-STJ4はSTJ検出器信号増幅可能であることが実証された。
	FD-SOI-MOSFETを用いた前置増幅器が極低温環境下(300mK環境下)で信号読み出し可能であることから、我々が関わる素粒子実験分野に留まらず、他分野への応用も期待される。
	
	今後は、本実験でのSOI-STJ4に伝送された信号が何光子相当であったか等、より定量的な解析を行い、性能評価をする必要がある。
	