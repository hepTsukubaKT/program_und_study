% ■ アブストラクトの出力 ■
%	◆書式:
%		begin{jabstract}〜end{jabstract}	:日本語のアブストラクト
%		begin{eabstract}〜end{eabstract}	:英語のアブストラクト
%		※ 不要ならばコマンドごと消せば出力されない。

% 日本語のアブストラクト
\begin{jabstract}
本研究では,浜松ホトニクス社が製造した内部増幅機能付き半導体粒子検出器(LGAD,Low Gain Avalanche Detector)の構造理解を含めた基礎特性の評価を行なった.IV特性の波長依存性から増幅層の厚さが3-20~$\mu$m程度であると推定し,CV特性から空乏化は電極とバルク部の境界,増幅層との境界,バルク部全体の順で進むことを推察した.また,ピクセル型センサーに$\gamma$線照射と中性子線照射を行い,照射前後でのIV特性の変化から放射線耐性を評価した.2.5~MGyまでの$\gamma$線照射における表面損傷では,LGADの増幅機能に大きな影響を与えないことが分かった.中性子線照射におけるバルク損傷では,増幅層の不純物濃度が高いほど,増幅機能を保持できることが分かった.特に,厚さの薄いサンプルでは,1.0$\times10^{15}$~n$_{eq}$/cm$^{2}$の照射後,700~V以下の逆バイアス電圧で10倍の増幅機能を保持していた.$3.0\times10^{15}$~n$_{eq}$/cm$^{2}$の照射後においても,LGADの構造が失われていないことを示唆するIV特性が見られた.今後,ストリップ型センサーに対して,レーザー・$\beta$線測定による増幅率の評価と,時間分解能の測定を行なっていく.
\end{jabstract}
