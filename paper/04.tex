%----------------------------------------------------------------------%
\chapter{LGADの放射線耐性評価}
\label{chap:tolerance}
%----------------------------------------------------------------------%
\section{$\boldsymbol\gamma$線照射}
ピクセル型LGADの表面損傷を評価するため,2016年11月に国立研究開発法人量子科学技術研究開発機構の高崎量子応用研究所にて,$\gamma$線照射を行った.
 %----------------------------------------------------------------------%
\subsection{$^{60}$Co$\gamma$線照射環境と照射量}
高崎量子応用研究所は,コバルト第1棟・コバルト第2棟・食品照射棟の3つの$\gamma$線照射棟を有する国内最大の研究用照射施設である.今回使用したのは食品照射棟第2照射室で,$^{60}$Co線源からの$\gamma$線を遮蔽する厚さ1.3 mの重コンクリートの壁に囲まれている(図\ref{fig:Taka}(左)).$^{60}$Co線源がプールから照射室内へ上昇し,照射が行われる.$^{60}$Co線源は,天然物質の$^{59}$Coに原子炉で中性子を照射することで作られるが,$\beta$崩壊することでさらにNiに変化し,その際2つの$\gamma$線(1.173MeV, 1.333MeV)を放出する(図\ref{fig:Taka}(右)).\par
\begin{figure}[H]
\begin{minipage}{0.5\hsize}
	\centering
	\includegraphics[width=75mm]{./pdf/Ta.pdf}
 \end{minipage}
 \begin{minipage}{0.5\hsize}
	\centering
	\includegraphics[width=50mm]{./pdf/ka.pdf}
 \end{minipage}
 	\caption{食品照射棟\cite{gamma}(左)と$^{60}$Coの崩壊図(右).}
	\label{fig:Taka}
\end{figure}
測定当時の食品照射棟第2照射室における線量率分布は図\ref{fig:sa}に示した通りであり,これより照射量に応じて,照射レートと照射時間を表\ref{tab:ki}のように決定した.また表\ref{tab:takasam}には,照射を行ったサンプルを$\circ$で示し,サンプル数が十分になかったため,照射を行えなかったサンプルは$\times$,実験の過程でスパークしたサンプルは{\scriptsize$\triangle$}で示してある.
\begin{figure}[H]
	\centering
	\includegraphics[width=140mm]{./pdf/sa.pdf}
	\caption{食品照射棟第2照射室の線量分布.}
	\label{fig:sa}
\end{figure}
\begin{table}[H]
	\centering
	\caption{照射レートと照射時間}
	\vspace{5truemm}
	\begin{tabular}{@{\hspace{0.5cm}}c@{\hspace{1cm}}c@{\hspace{1cm}}c@{\hspace{0.5cm}}}\hline
	\textbf{Dose}& \textbf{Rate}& \textbf{Time}\\
	\hline\hline
	0.1MGy& 5kGy/h& 23h\\
	\hline
	1.0MGy& 2kGy/h& 500h\\
	\hline
	2.5MGy& 5kGy/h& 500h\\
	\hline
	\end{tabular}
	\label{tab:ki}
\end{table}
\begin{table}[H]
	\centering
	\caption{ピクセル型センサーの$\gamma$線照射サンプル一覧.}
	\vspace{5truemm}
	\begin{tabular}{|c||c|c|c|c|c|c|c|c|}\hline
	\multirow{2}{*}{\textbf{Gamma Dose}}& \multicolumn{8}{|c|}{\textbf{Sample Name}}\\ \cline{2-9}
	&50A &50B &50C &50D &80A &80B &80C &80D\\ \cline{2-9}
	\hline\hline
	0.1MGy& $\times$& {\scriptsize$\triangle$}& $\circ$& $\circ$& {\scriptsize$\triangle$}& {\scriptsize$\triangle$}& {\scriptsize$\triangle$}& $\circ$\\
	\hline
	1.0MGy& {\scriptsize$\triangle$}& $\circ$& $\circ$& $\circ$& $\times$& $\circ$& $\times$& $\circ$\\
	\hline
	2.5MGy& $\times$& $\circ$& $\circ$& $\circ$& $\circ$& $\circ$& $\circ$& $\circ$\\
	\hline
	\end{tabular}
	\label{tab:takasam}
\end{table}
%----------------------------------------------------------------------%
\subsection{暗電流とLED光応答測定}
$\gamma$線照射前後での暗電流とLED赤外光($\lambda\sim$850~nm)応答の変化を図\ref{fig:G}に示す.いずれのサンプルにおいても暗電流は1桁程度増加しているが,暗電流の増加量は増幅層の不純物濃度には依存しないことが分かった.これは,第\ref{sub:damage}節で前述した表面損傷によるものだと考えられる.いずれのサンプルにおいても,照射前後のLED光応答に大きな差は見られない.LGADの増幅機能は表面損傷では失われないことが確かめられた.
\begin{figure}[H]
\begin{minipage}{0.5\hsize}
	\centering
	\includegraphics[width=75mm]{./graph/G50B.pdf}
\end{minipage}
\begin{minipage}{0.5\hsize}
	\centering
	\includegraphics[width=75mm]{./graph/G80B.pdf}
\end{minipage}
\begin{minipage}{0.5\hsize}
	\centering
	\includegraphics[width=75mm]{./graph/G50C.pdf}
\end{minipage}
\begin{minipage}{0.5\hsize}
	\centering
	\includegraphics[width=75mm]{./graph/G80C.pdf}
\end{minipage}
\begin{minipage}{0.5\hsize}
	\centering
	\includegraphics[width=75mm]{./graph/G50D.pdf}
\end{minipage}
\begin{minipage}{0.5\hsize}
	\centering
	\includegraphics[width=75mm]{./graph/G80D.pdf}
\end{minipage}
 	\caption{$\gamma$線照射前後の暗電流変化.}
	\label{fig:G}
\end{figure}
%----------------------------------------------------------------------%
\newpage
\section{中性子線照射}
ピクセル型LGADのバルク損傷を評価するため,2016年12月にSloveniaのLjubljanaにあるヨージェフ・ステファン研究所にて,中性子線照射を行ったサンプルを用いて測定を行った.\par
\subsection{中性子線照射環境と照射量}
Ljubljanaへ発送するための準備を2016年12月8日に高エネルギー加速器研究機構で行い,翌9日に発送した.12月14日にLjubljanaに到着したサンプルは,翌15日に
%現地にいるIgor Mardic,Vladimir Cindro両名によって
中性子線照射された.Ljubljanaからは2017年1月3日に発送され,6日に高エネルギー加速器研究機構に到着するまでの3日間は0~℃に保たれていた.それ以降は,作業時と測定時を除いて$-$20~℃で管理した.\par
照射量と照射したサンプルの一覧を表\ref{tab:neusam}に示す.ただし,{\scriptsize$\triangle$}は実験の過程でスパークしたサンプルである.50Aのサンプルについては測定が終わっていないため空欄になっている.
\begin{table}[H]
	\centering
	\caption{ピクセル型センサーの中性子線照射サンプル一覧.}
	\vspace{5truemm}
	\begin{tabular}{|c||c|c|c|c|c|c|c|c|}\hline
	\multirow{2}{*}{\textbf{Neutron Fluence}}& \multicolumn{8}{|c|}{\textbf{Sample Name}}\\ \cline{2-9}
	&50A &50B &50C &50D &80A &80B &80C &80D\\ \cline{2-9}
	\hline\hline
	0.3$\times$10$^{15}$~n$_{eq}$/cm$^{2}$& & {\scriptsize$\triangle$}& {\scriptsize$\triangle$}& $\circ$& {\scriptsize$\triangle$}& {\scriptsize$\triangle$}& {\scriptsize$\triangle$}& $\circ$\\
	\hline
	1.0$\times$10$^{15}$~n$_{eq}$/cm$^{2}$& & $\circ$& $\circ$& $\circ$& {\scriptsize$\triangle$}& $\circ$& $\circ$& $\circ$\\
	\hline
	3.0$\times$10$^{15}$~n$_{eq}$/cm$^{2}$& & $\circ$& $\circ$& $\circ$& $\circ$& $\circ$& $\circ$& $\circ$\\
	\hline
	\end{tabular}
	\label{tab:neusam}
\end{table}
%----------------------------------------------------------------------%
\subsection{暗電流とLED光応答測定}
中性子線照射後のサンプルの電流を照射量で比較するためには,60~℃で80分アニーリングさせる必要がある\cite{moll}.そこで測定は60~℃で80分のアニーリング後に行なった.また,LGADの増幅機能が残っているかを判断しやすいように,第\ref{sub:temp}項で示した温度に関する依存性と第\ref{sub:wave}項で示した波長に関する依存性から,より低い温度($-$40~℃)で,より長い波長($\sim$850~nm)の赤外LEDを用いて測定を行なった.中性子線照射前後での暗電流とLED赤外光応答の変化を図\ref{fig:N}に示す(他のサンプルについては付録\ref{neuN}参照).第\ref{sub:damage}節で前述したように,照射量に伴い暗電流は増加するはずであるが,0.3$\times$10$^{15}$~n$_{eq}$/cm$^{2}$の照射量に関しては,明らかに他の照射量に比べて大きい.50C・50D・80Dいずれのサンプルにおいても同程度に大きくなっていることから,表面の汚れ等による増加とは考え難い.50C・50Dのサンプルでは500~V以降で1.0$\times$10$^{15}$~n$_{eq}$/cm$^{2}$が3.0$\times$10$^{15}$~n$_{eq}$/cm$^{2}$と同程度まで増加しているが,これは低い照射量の方がLGADの増幅機能が低電圧においても十分残っているために,暗電流の増幅が見えていると解釈できる.
LED光応答では,照射量が多いほど増幅機能が見える電圧値が高くなっているが,これは従来型のN$^{+}$-in-Pセンサーでも中性子線照射時のバルク比抵抗の低下に伴い見られる高電圧へのシフトである.特にActive Thicknessの薄い50C・50Dについては,未照射時の増幅が見られる電圧値が低いために,高照射後でもLGADの増幅機能が残っていることがIV特性からも分かる.一方,Active Thicknessの厚いサンプルでは,照射量1.0$\times$10$^{15}$~n$_{eq}$/cm$^{2}$を越えるとはっきりと増幅が見られるものはないが,高電圧において電流値の上昇は見ることができる.増幅率の変化を表\ref{tab:gain}にまとめた.ただし,$-$100~Vでの電流値を基準とした.\par
\begin{table}[H]
	\centering
	\caption{増幅率が10となる電圧値の変化.}
	\vspace{5truemm}
	\begin{tabular}{|c||c|c|c|c|}\hline
	\multirow{2}{*}{\textbf{Sample}}& \multicolumn{4}{|c|}{\textbf{Neutron Fluence}}\\ \cline{2-5}
	&Non Irradiated &0.3$\times$10$^{15}$~n$_{eq}$/cm$^{2}$ &1.0$\times$10$^{15}$~n$_{eq}$/cm$^{2}$ &3.0$\times$10$^{15}$~n$_{eq}$/cm$^{2}$\\ \cline{2-5}
	\hline\hline
	50C& 380V& 550V& 700V& $\textgreater$700V\\
	\hline
	50D& 260V& 470V& 640V& $\textgreater$700V\\
	\hline
%	80A& 610V& & & $\textgreater$800V\\
%	\hline
%	80B& 630V& & $\textgreater$800V& $\textgreater$800V\\
%	\hline
	80C& 540V& & $\textgreater$800V& $\textgreater$800V\\
	\hline
	80D& 350V& 760V& $\textgreater$800V& $\textgreater$800V\\
	\hline
	\end{tabular}
	\label{tab:gain}
\end{table}
未照射時の増幅が低電圧で見られる方が,放射線照射後も低電圧で増幅が見られることを確認した.つまり,増幅層の不純物濃度が高く,Active Thicknessの薄いほど,放射線耐性も高いと言える.\par
\begin{figure}[H]
\begin{minipage}{0.55\hsize}
	\centering
	\includegraphics[width=75mm]{./graph/NLE50C.pdf}
\end{minipage}
\begin{minipage}{0.45\hsize}
	\centering
	\includegraphics[width=75mm]{./graph/NLI50C.pdf}
\end{minipage}
\begin{minipage}{0.55\hsize}
	\centering
	\includegraphics[width=75mm]{./graph/NLE50D.pdf}
\end{minipage}
\begin{minipage}{0.45\hsize}
	\centering
	\includegraphics[width=75mm]{./graph/NLI50D.pdf}
\end{minipage}
\begin{minipage}{0.55\hsize}
	\centering
	\includegraphics[width=75mm]{./graph/NLE80C.pdf}
\end{minipage}
\begin{minipage}{0.45\hsize}
	\centering
	\includegraphics[width=75mm]{./graph/NLI80C.pdf}
\end{minipage}
\begin{minipage}{0.55\hsize}
	\centering
	\includegraphics[width=75mm]{./graph/NLE80D.pdf}
\end{minipage}
\begin{minipage}{0.45\hsize}
	\centering
	\includegraphics[width=75mm]{./graph/NLI80D.pdf}
\end{minipage}
 	\caption{中性子線照射前後のIV特性変化(暗電流:左,光応答:右).}
	\label{fig:N}
\end{figure}