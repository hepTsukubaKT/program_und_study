%----------------------------------------------------------------------%
\chapter{LHCとATLAS}
\label{chap:start}
%----------------------------------------------------------------------%
\section{LHC}
LHC (Large Hadron Collider)は,欧州合同原子核研究機構CERNに建設され,2009年より運転を開始した世界最大のハドロン衝突型加速器である.この周長27kmにも及ぶ加速器は,スイスとフランスの国境ジュネーブ近郊,地下約100mに設置されている(図\ref{fig:LHC}).\par
\begin{figure}[h]
	\centering
	\includegraphics[width=120mm]{./pdf/LHC.pdf}
	\caption{LHC加速器の外観\cite{cern}.}
	\label{fig:LHC}
\end{figure}
CERNの加速器のビームは,小さな加速器から大きな加速器へと次々にエネルギーを上げながらLHCまで運ばれる(図\ref{fig:LHCs}).LHCでは,ビームは超高真空に保たれた2本のビームパイプ内を逆方向に進む.超電導磁石によって作られた高磁場が,7~TeVの陽子ビームを保持し,ビームを収束させることで,衝突点で素粒子反応を発生する.\par
\begin{figure}[h]
	\centering
	\includegraphics[width=150mm]{./pdf/LHCs.pdf}
	\caption{CERNの加速器群とLHC\cite{cern}.}
	\label{fig:LHCs}
\end{figure}
LHCには4つの大きな実験グループがあり,衝突点にはそれぞれの目的に合わせたALICE (A Large Ion Collider Experiment),ATLAS (A Toroidal LHC ApparatuS),CMS (Compact Muon Solenoid),LHCb (LHC beauty)と呼ばれる検出器を設置している.ALICEは鉛の原子核-原子核衝突によって高温状態を作り出すことで,粒子がクォーク・グルーオンプラズマからどのように生成されたかを研究する.ATLASとCMSは,陽子-陽子衝突によって発生した粒子の解析を行うことで,質量の起源や暗黒物質・暗黒エネルギーなどの解明を目指している.LHCbはbクォークの精密測定を目的としているが,LHCには他にもより限られた目的を持つLHCf (LHC forward),
TOTEM (TOTal Elastic and diffractive cross-section measurement),MoEDAL (Monopole and Exotics Detector at LHC)と呼ばれる検出器もあり,LHCは様々な物理に対して今後も期待されている.\par
2026年に開始するLHCの高輝度化計画(HL-LHC計画)では,入射器や収束磁石の改良により,瞬間輝度が現在の5倍以上に増強される予定である.これに伴い,検出器が置かれる放射線環境はより厳しくなるため,実現に向けて多くの関連分野で開発が進められている.
%----------------------------------------------------------------------%
\section{ATLAS}
ATLAS (A Toroidal LHC ApparatuS)は,LHCの衝突点のひとつに設置された汎用型粒子検出器である.全長44~m,高さ25~m,重量7000~tの大型検出器であるが,複数の検出器が組み合わされている(図\ref{fig:ATLAS}).内側から内部飛跡検出器,電磁カロリメータ,ハドロンカロリメータ,ミュー粒子検出器が設置されている.超電導ソレノイド磁石によって作られる2~Tの磁場中に置かれた内部飛跡検出器は,磁場で曲げられた荷電粒子の飛跡再構成や運動量測定を行う.電磁カロリメータは,電子・光子のエネルギー測定や電磁シャワーの位置測定を行う.ハドロンカロリメータはジェットのエネルギー測定を行い,トロイダル磁石によって作られる磁場中に置かれたミュー粒子検出器はミュー粒子の運動量測定を行う.
\begin{figure}[H]
	\centering
	\includegraphics[width=120mm]{./pdf/ATLAS.pdf}
	\caption{ATLAS\cite{cern}.}
	\label{fig:ATLAS}
\end{figure}
%----------------------------------------------------------------------%
\subsection*{内部飛跡検出器}
内部飛跡検出器は,全長5.3~m,直径2.1~mであり,ATLASの最も内側に位置する.超電導ソレノイド磁石が作る
2~Tの磁場によって曲げられた荷電粒子の曲率半径を求めることで,粒子の運動量測定と飛跡再構成による崩壊点測定を行う.現在のATLASでは内側からピクセル検出器,マイクロストリップ検出器,ストローチューブ型遷移輻射検出器という構造がとられている(図\ref{fig:ITK}).しかしHL-LHC計画では,ガス検出器であるストローチューブ型遷移輻射検出器は廃止され,全ての検出器が新しいシリコン検出器に置き換わることが決定している.これに向けて,放射線耐性に優れたセンサーの開発が進められている.
\begin{figure}[H]
	\centering
	\includegraphics[width=120mm]{./pdf/ITK.pdf}
	\caption{内部飛跡検出器\cite{cern}.}
	\label{fig:ITK}
\end{figure}
%----------------------------------------------------------------------%








