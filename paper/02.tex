%----------------------------------------------------------------------%
\chapter{半導体検出器}
\label{chap:semicon}
%----------------------------------------------------------------------%
\section{動作原理}
%----------------------------------------------------------------------%
\subsection{半導体}
通常原子が単独で存在する場合と,結晶中のように複数存在する場合では,電子のエネルギー状態が異なる.シリコンでは,原子同士が接近すると,$3s$, $3p$準位は分裂してエネルギー帯を形成する (図\ref{fig:Si}).電子の占有が許される領域を許容帯(allowed band)と呼ばれ,低い方のエネルギー帯である価電子帯(valence band)と,高い方のエネルギー帯である伝導帯(conduction band)に分かれる.一方これらの間に挟まれるエネルギー領域は,電子の準位が存在しない禁制帯(forbidden band)であり,このエネルギー幅を禁制帯幅(energy gap,$E_{g}$)と呼ぶ.\par
\begin{figure}[h]
	\centering
	\includegraphics[width=120mm]{./pdf/Si.pdf}
	\caption{シリコンのエネルギー帯構造.}
	\label{fig:Si}
\end{figure}
物質の導電性はエネルギー帯構造によって説明することができる.絶縁体は禁制帯幅が大きく,室温では価電子帯から伝導帯に電子が移ることはできないので,電流は流れにくい.一方導体は,価電子帯と伝導帯のエネルギー帯が重なっているため,電子は自由に移動することができ,電流は流れやすくなる.半導体の場合は,価電子帯と伝導帯が離れてはいるが,絶縁体に比べて禁制帯幅が小さいので,室温における原子の熱振動による熱エネルギーでも,電子が伝導帯に移ることができる (図\ref{fig:InSeCo}).\par
\begin{figure}[h]
	\centering
	\includegraphics[width=120mm]{./pdf/InSeCo.pdf}
	\caption{絶縁体・半導体・導体のエネルギー帯.}
	\label{fig:InSeCo}
\end{figure}
%----------------------------------------------------------------------%
\subsection{半導体の種類}
不純物のない半導体は真性半導体(intrinsic semiconductor)と呼ばれる.真性半導体にキャリアとして存在するのは,熱励起によって価電子帯から伝導帯に移った電子と,価電子帯に残された正孔(hole)のみである.そのため伝導体にある電子の数と,価電子帯にある正孔の数は必ず一致する (図\ref{fig:Thermal}).\par
\begin{figure}[h]
	\centering
	\includegraphics[width=100mm]{./pdf/Thermal.pdf}
	\caption{熱励起.}
	\label{fig:Thermal}
\end{figure}
真性半導体はキャリア密度が温度変化に対して敏感に変化するため,密度を制御することが難しく,デバイスに利用されることは少ない.このため実用的には,真性半導体にある種類の不純物元素を添加することにより,キャリアの一方のみを増加させるなど,半導体の抵抗率を自由に制御する手法が用いられる.このような不純物を含んだ半導体を不純物半導体(extrinsic semiconductor)という.不純物半導体の中で,密度の高い方のキャリアを多数キャリア,低い方のキャリアを少数キャリアと呼び,多数キャリアが電子である半導体をN型半導体,多数キャリアが正孔である半導体をP型半導体という.\par

\subsubsection*{N型半導体}
シリコン結晶に5族の不純物元素(P, As等)を添加すると,格子点上のシリコン原子と置換することで結晶格子の一部となる.5価の原子が持つ5個の価電子の内,4個は隣接するシリコン原子との共有結合に用いられるが,残りの1個は結合には関与せず,5価の原子にゆるく結合されている.新たに作られたこの準位はドナー準位(donor level, $E_{D}$)と呼ばれ,比較的小さなイオン化エネルギー($E_{C}-E_{D}$)のみで電子は伝導帯に移ることができる (図\ref{fig:Ntype}). 室温ではほとんどの電子が伝導帯に移るため,1個の5族元素は,1個の電子キャリアを結晶に与えることになる.そのためこの5族元素をドナーという.ドナーが添加された半導体はキャリアが負電荷(negative charge)になることから,N型半導体と呼ばれている.
\begin{figure}[h]
	\centering
	\includegraphics[width=150mm]{./pdf/Ntype.pdf}
	\caption{N型半導体の結晶構造とエネルギー帯.}
	\label{fig:Ntype}
\end{figure}

\subsubsection*{P型半導体}
シリコン結晶に3族の不純物元素(B等)を添加すると,格子上のシリコン原子と置換することで結晶格子の一部となる.3価の原子が持つ3個の価電子は,隣接するシリコン原子3個との共有結合に用いられる.さらにもう1個の共有結合のために周囲の価電子を1個受け取る.そのためこの3族元素をアクセプタという.受け取る際に,結晶中に1個の正孔を放出する.新たに作られたこの準位はアクセプタ準位と呼ばれ,室温ではほとんどのアクセプタがイオン化し,価電子帯に正孔を放出する.アクセプタが添加された半導体はキャリアが正電荷(positive charge)になることから,P型半導体と呼ばれている.
\begin{figure}[h]
	\centering
	\includegraphics[width=150mm]{./pdf/Ptype.pdf}
	\caption{P型半導体の結晶構造とエネルギー帯.}
	\label{fig:Ptype}
\end{figure}
%----------------------------------------------------------------------%
\subsection{PN接合}
\label{sub:pn}
P型半導体とN型半導体が接合すると,接合部を境にキャリアに大きな密度差が生じる.N型半導体の電子はP型領域に拡散し,正孔と再結合して消滅する.逆にP型領域の正孔はN型領域へと拡散し,電子と再結合する.一方,ドナーイオンやアクセプタイオンは格子上から動くことができないため,P型領域は負,N型領域は正の空間電荷が生じる.この領域は空乏層(Depletion Region)と呼ばれ,キャリアはほとんど存在しない (図\ref{fig:NoBias}).\par
\begin{figure}[h]
	\centering
	\includegraphics[width=95mm]{./pdf/NoBias.pdf}
	\includegraphics[width=90mm]{./pdf/name.pdf}
	\caption{PN接合のエネルギー帯.}
	\label{fig:NoBias}
\end{figure}
この空間電荷による電場は,拡散電位という電位差$V_{d}$を生じさせる.これと準位の関係は$E_{CP}-E_{CN}=eV_{d}$で書ける.また,電場の向きは,N型領域からP型領域であり,この電場によるドリフトの向きは,キャリア密度差による拡散の向きと逆向きになる.この2つの働きが釣り合うのが, PN接合の熱平衡状態である.\par

\subsubsection*{順方向バイアス}
P型領域側が正,N型領域側が負となるように電圧$V_{F}$を加えると,その分,P型領域に対するN型領域の電子のエネルギーは相対的に高くなる (図\ref{fig:FoBias}).つまり障壁の高さが低くなる($E_{CP}-E_{CN}=e(V_{d}-V_{F})$)ので,P型領域の多数キャリアである正孔は障壁を越えてN型領域へ,N型領域の多数キャリアである電子も障壁を越えてP型領域へ拡散により移動でき,電流が流れるようになる.この方向に加える外部電圧を順方向バイアスという.\par
\begin{figure}[h]
	\centering
	\includegraphics[width=84mm]{./pdf/FoBias.pdf}
	\caption{PN接合の順方向バイアス印加時のエネルギー帯.}
	\label{fig:FoBias}
\end{figure}

\subsubsection*{逆方向バイアス}
P型領域側が負,N型領域側が正となるように電圧$V_{R}$を加えると,その分,P型領域に対するN型領域の電子のエネルギーは相対的に低くなる (図\ref{fig:ReBias}).つまり障壁の高さが高くなる($E_{CP}-E_{CN}=e(V_{d}+V_{R})$)ので,多数キャリアは他方の領域へ拡散により移動することができない.この方向に加える外部電圧を逆方向バイアスという.空乏層端の少数キャリアは,電位障壁を降り他方に移動できるが,その密度は逆方向バイアスに依存せず一定なので,ほとんど電流は流れない.\par
\begin{figure}[h]
	\centering
	\includegraphics[width=84mm]{./pdf/ReBias.pdf}
	\caption{PN接合の逆方向バイアス印加時のエネルギー帯.}
	\label{fig:ReBias}
\end{figure}
%----------------------------------------------------------------------%
\subsection{半導体検出器}
半導体検出器はPN接合を基礎とする.PN接合の空乏層に荷電粒子や十分なエネルギーを持つ光子が入射すると,そのエネルギーにより電子正孔対が生成される.この電子や正孔を信号として収集するのが半導体検出器である.空乏層は厚い方が生成される電子正孔対は増加するため, 逆方向バイアスで動作させる.逆方向バイアスは,電子と正孔を電極側により引き寄せるため,信号を効率的に収集するという意味においても適している.図\ref{fig:Detector}に典型的なストリップ型半導体検出器の構造を示す.大きく分けてバルクにN型を用いて電極をP$^{+}$にするP$^{+}$-in-N型センサーと,バルクにP型を用いて電極をN$^{+}$にするN$^{+}$-in-P型センサーがあるが,現在HL-LHC計画に向けて開発されているのは,N$^{+}$-in-P型センサーである.\par
\begin{figure}[h]
	\centering
	\includegraphics[width=65mm]{./pdf/DetectorN.pdf}
	\includegraphics[width=65mm]{./pdf/DetectorP.pdf}
	\caption{典型的な半導体検出器の構造. N型バルク(左)とP型バルク(右).}
	\label{fig:Detector}
\end{figure}
%----------------------------------------------------------------------%
\section{半導体検出器の特性}
%----------------------------------------------------------------------%
\subsection{IV特性}
\label{sub:iviv}
PN接合に順方向バイアス$V_{F}$を加えたときに流れる電流密度$J$は,次式で表される.
\begin{eqnarray*}
J=e(\frac{D_{p}p_{n0}}{L_{p}}+\frac{D_{n}n_{p0}}{L_{n}})({\rm exp}(\frac{eV_{F}}{k_{B}T})-1)
\end{eqnarray*}
\begin{eqnarray*}
D_{n,p}=電子あるいは正孔の拡散定数,L_{n,p}=電子あるいは正孔の拡散距離,\\
n_{p0}=P型領域の電子密度,p_{n0}=N型領域の正孔密度,k_{B}=ボルツマン定数
\end{eqnarray*}
$V_{F}$が大きくなると指数項の寄与が大きくなるため,順方向バイアスにおける電流は指数関数的に増加する.逆方向バイアスに対しても$J$は同様に表されるが,$V_{R}$が大きくなると指数項が1に対して無視できる程度に小さくなるため,逆方向バイアスでは小さい一定の電流が流れるようになる.半導体検出器は逆方向バイアスで動作するため,センサーにはほぼ一定の小さな電流が流れる.しかし,逆方向バイアスをさらに大きくしていくと,ある電圧で急激に大量の電流が流れる.このときの電圧を降伏電圧(breakdown voltage)と呼ぶ.以上よりIV特性の概略は図\ref{fig:IVbreak}のようになる.\par
\begin{figure}[H]
	\centering
	\includegraphics[width=75mm]{./pdf/breakdown.pdf}
	\caption{PN接合のIV特性概略図.}
	\label{fig:IVbreak}
\end{figure}
降伏電圧による大電流を流し続けるとセンサーを熱的に破壊してしまうため,印加電圧はこの降伏電圧より小さい必要がある.降伏の機構にはツェナー降伏(Zener breakdown)と雪崩降伏(avalanche breakdown,アバランシェ降伏)がある.ツェナー降伏は逆方向バイアスによってPN接合における価電子帯と伝導帯がの空間的な距離が縮まり,量子力学的なトンネル効果によって,価電子帯の電子が伝導帯へ直接通り抜ける現象である(図\ref{fig:break}左)).雪崩降伏は,空乏層の電場によって加速されたキャリアが,結晶格子上の原子内の価電子に衝突することで電子正孔対を生成し,さらにその生成された電子が同じことを繰り返していくことで起こる(図\ref{fig:break}(右)).一般に温度が高くなると禁制帯幅$E_{g}$が小さくなるため,トンネル現象が生じやすく,ツェナー降伏はより低電圧で起こるようになる.一方,温度の増大に伴い格子振動が激しくなるため,キャリアの移動度は小さくなり,アバランシェ降伏はより高電圧で起こるようになる.
\begin{figure}[H]
	\centering
	\includegraphics[width=100mm]{./pdf/break.pdf}
	\caption{ツェナー降伏(左)と雪崩降伏(右).}
	\label{fig:break}
\end{figure}
%----------------------------------------------------------------------%
\subsection{CV特性}
\label{sub:cv}
PN接合における空間電荷の分布と電場・電位を図\ref{fig:cvcv}に示す.ただし,接合面を$x=0$とし,$x$が負の領域をアクセプタ密度が一定のP型領域,$x$が正の領域をドナー密度が一定のN型領域であると仮定した.さらに,空乏層内には不純物イオンによる空間電荷が存在し,空乏層外では不純物イオンとキャリアによって電気的に中性であるという空乏近似を用いると,空乏層厚$w$は次式で表すことができる.
\begin{eqnarray*}
w=x_{p}+x_{n}=\sqrt{\frac{2\epsilon_{0}\epsilon_{Si}(N_{a}+N_{d})}{eN_{a}N_{d}}(V_{d}+V_{R})}
\end{eqnarray*}
\begin{eqnarray*}
N_{a,d}=アクセプターあるいはドナー密度,\epsilon_{0}=真空の誘電率,\epsilon_{Si}=シリコンの比誘電率
\end{eqnarray*}
N$^{+}$-in-P型センサーにおける電極とバルク部の空乏化では,$N_{a}\gg N_{d}$であり,式からバルク部の$N_{d}$が小さいほど同じ電圧での空房層厚は厚くなることが分かる.
空乏層の単位面積当たりの静電容量は,次式で表される.
\begin{eqnarray*}
C=\frac{\epsilon_{0}\epsilon_{Si}}{w}=\sqrt{\frac{\epsilon_{0}\epsilon_{Si}eN_{a}N_{d}}{2(N_{a}+N_{d})}\frac{1}{(V_{d}+V_{R})}}
\end{eqnarray*}
これより,CV特性の概略は図\ref{fig:cvcvcv}のようになる.また,このCV特性の傾きは,
\begin{eqnarray*}
\frac{dC}{dV_{R}}=-\frac{1}{2}\frac{C}{V_{d}+V_{R}}
\end{eqnarray*}
であるので,N$^{+}$-in-P型センサーにおける電極とバルク部の空乏化では,バルク部の$N_{d}$が小さいほど静電容量の電圧に対する変化率は大きくなる.
\begin{figure}[h]
\begin{minipage}[b]{0.6\hsize}
	\centering
	\includegraphics[width=100mm]{./pdf/cvcv.pdf}
	\caption{PN接合の空間電荷分布と電場・電位.}
	\label{fig:cvcv}
\end{minipage}
\begin{minipage}[b]{0.4\hsize}
	\centering
	\includegraphics[width=60mm]{./pdf/capa.pdf}
	\caption{PN接合のCV特性概略図.}
	\label{fig:cvcvcv}
\end{minipage}
\end{figure}
%----------------------------------------------------------------------%
\subsection{放射線損傷}
\label{sub:damage}
半導体検出器の放射線損傷は,大きく表面損傷とバルク損傷に分けられる(図\ref{fig:damageS}).
\subsubsection*{表面損傷}
放射線が照射されると,バルク部の空乏層だけでなく,SiO$_{2}$層においても電子正孔対が生成される.電子と正孔では移動度が異なるため,移動度の小さい正孔のみがSiO$_{2}$層にトラップされてしまう.これが蓄積していくことで,SiO$_{2}$層は正に帯電する.このSiO$_{2}$層の電荷は,センサーの性能に影響を与え,暗電流を増加させたり,表面電場の上昇による降伏電圧の変化を引き起こす.この損傷はバルク部では起こらず,酸化膜のある表面でのみ起こるため表面損傷と呼ばれる.蓄積する電荷量には限界があるので,一般的に表面損傷は線量増大に対して飽和する傾向がある.
\subsubsection*{バルク損傷}
センサーに放射線が入射すると,電子の励起や,結晶格子上の原子との衝突などによりエネルギーを失う.ここで,衝突により原子が結晶表面まで弾き出され,原子空孔(vacancy)のみが残されたものをショットキー欠陥と呼び,結晶表面までは移動せずに結晶格子間で止まった場合,その格子間原子と原子空孔の対をフレンケル欠陥と呼ぶ.これらの格子欠陥は,結晶に新たな不純物準位を形成するため,電子・正孔がトラップされることによる収集電荷量の低下や,暗電流の増加を招く.これら格子欠陥はP型やN型など種々に振舞うが,シリコンの場合は統合的にはP型の不純物の生成の方が多い.熱中性子が衝突すると,周囲の格子が集団で破壊されることも多い.そのため,中性子による損傷が陽子による損傷よりも大きいことが分かっている.
\begin{figure}[h]
	\centering
	\includegraphics[width=100mm]{./pdf/damageS.pdf}
	\caption{表面損傷(左)とバルク損傷(右).}
	\label{fig:damageS}
\end{figure}









