% 独自のコマンド

% ■ アブストラクト
%  \begin{jabstract} 〜 \end{jabstract}  :日本語のアブストラクト
%  \begin{eabstract} 〜 \end{eabstract}  :英語のアブストラクト

% ■ 謝辞
%  \begin{acknowledgment} 〜 \end{acknowledgment}

% ■ 文献リスト
%  \begin{bib}[100] 〜 \end{bib}


\newif\ifjapanese

\japanesetrue	% 論文全体を日本語で書く

\ifjapanese
%  \documentclass[a4j,11pt,oneside,openany]{ujreport}
  \documentclass[a4j,11pt,twoside]{ujreport}
  \renewcommand{\bibname}{参考文献}
  \newcommand{\acknowledgmentname}{謝辞}
\else
  \documentclass[a4paper,11pt]{report}
  \newcommand{\acknowledgmentname}{Acknowledgment}
\fi

%----------------------------------------------------------------------%
%章番号の後に改行を入れたくない時
%\makeatletter
%\def\@makechapterhead#1{
%\vspace*{2\Cvs}
%{\parindent \z@ \raggedright \normalfont
%\Huge\headfont
%\ifnum \c@secnumdepth >\m@ne
%\if@mainmatter
%\@chapapp\thechapter\@chappos
%\hskip1zw
%\fi
%\fi
%#1\par\nobreak
%\vskip 3\Cvs}}
%\makeatother
%----------------------------------------------------------------------%

%----------------------------------------------------------------------%
%表と図を並べる時
%\makeatletter
%\newcommand{\figcaption}[1]{\def\@captype{figure}\caption{#1}}
%\newcommand{\tblcaption}[1]{\def\@captype{table}\caption{#1}}
%\makeatother
%本文内
%\begin{figure}[h]
%\begin{tabular}{cc}
%\begin{minipage}[b]{.70\textwidth}
%	\begin{center}
%	\includegraphics[]{}
%	\end{center}
%	\figcaption{}
%	\label{}
%\end{minipage}
%\begin{minipage}[b]{.25\textwidth}
%	\begin{center}
%	\tblcaption{}
%	\begin{tabular}{ccc}\hline
%	\end{tabular}
%	\label{}
%	\end{center}
%\end{minipage}
%\end{tabular}
%\end{figure}
%----------------------------------------------------------------------%
%\documentclass{jsarticle}
\usepackage{thesis}
\usepackage{ascmac}
\usepackage[dvipdfmx]{graphicx}
\usepackage{multirow}
\usepackage{url}
\usepackage{float}
\usepackage{comment} 
\usepackage{plext}
\usepackage{amsmath}
\usepackage[jis2004,deluxe,uplatex]{otf}
%\usepackage{utf}


\bibliographystyle{jplain}

%\bindermode  % バインダー用余白設定

\setlength{\textwidth}{145truemm}		% テキスト幅: 210mm
\setlength{\oddsidemargin}{45truemm}	% 左余白
\addtolength{\oddsidemargin}{-1truein}	% 左位置デフォルトから-1inch
\setlength{\topmargin}{20truemm}		% 上余白
\setlength{\textheight}{220truemm}		% テキスト高さ: 297mm
\addtolength{\topmargin}{-1truein}		% 上位置デフォルトから-1inch

%----------------------------------------------------------------------%
\jclass  {卒業論文}				% 論文種別
\jtitle    {卒研終われ終われ同盟}	% タイトル
\juniv    {筑波大学}				% 大学名
\jfaculty  {理工学群物理学類}	% 学部、学科
\jauthor  {\CID{8705}橋\ 光太郎}							% 著者
\jhyear  {30}						% 平成○年度
\jsyear  {2018}					% 西暦○年度
\jkeyword  {LGAD, fast silicon sensors, thin tracking sensors, radiation hardness}		% 論文のキーワード
\jproject{指導教員\hspace{5mm}辻 悠汰\hspace{5mm}\small{印}}		%指導教員
\jdate{2019年2月}

\setcounter{topnumber}{3}
\setcounter{bottomnumber}{3}
\setcounter{totalnumber}{3}

%----------------------------------------------------------------------%
\begin{document}

\ifjapanese
  \jmaketitle    % 表紙(日本語)
\else
  \emaketitle    % 表紙(英語)
\fi

% ■ アブストラクトの出力 ■
%	◆書式:
%		begin{jabstract}〜end{jabstract}	:日本語のアブストラクト
%		begin{eabstract}〜end{eabstract}	:英語のアブストラクト
%		※ 不要ならばコマンドごと消せば出力されない。

% 日本語のアブストラクト
\begin{jabstract}
本研究では,浜松ホトニクス社が製造した内部増幅機能付き半導体粒子検出器(LGAD,Low Gain Avalanche Detector)の構造理解を含めた基礎特性の評価を行なった.IV特性の波長依存性から増幅層の厚さが3-20~$\mu$m程度であると推定し,CV特性から空乏化は電極とバルク部の境界,増幅層との境界,バルク部全体の順で進むことを推察した.また,ピクセル型センサーに$\gamma$線照射と中性子線照射を行い,照射前後でのIV特性の変化から放射線耐性を評価した.2.5~MGyまでの$\gamma$線照射における表面損傷では,LGADの増幅機能に大きな影響を与えないことが分かった.中性子線照射におけるバルク損傷では,増幅層の不純物濃度が高いほど,増幅機能を保持できることが分かった.特に,厚さの薄いサンプルでは,1.0$\times10^{15}$~n$_{eq}$/cm$^{2}$の照射後,700~V以下の逆バイアス電圧で10倍の増幅機能を保持していた.$3.0\times10^{15}$~n$_{eq}$/cm$^{2}$の照射後においても,LGADの構造が失われていないことを示唆するIV特性が見られた.今後,ストリップ型センサーに対して,レーザー・$\beta$線測定による増幅率の評価と,時間分解能の測定を行なっていく.
\end{jabstract}
	% アブストラクト。要独自コマンド、include先参照のこと

\tableofcontents
\listoffigures
\listoftables

\pagenumbering{arabic}

%%----------------------------------------------------------------------%
\chapter{LHCとATLAS}
\label{chap:start}
%----------------------------------------------------------------------%
\section{LHC}
LHC (Large Hadron Collider)は,欧州合同原子核研究機構CERNに建設され,2009年より運転を開始した世界最大のハドロン衝突型加速器である.この周長27kmにも及ぶ加速器は,スイスとフランスの国境ジュネーブ近郊,地下約100mに設置されている(図\ref{fig:LHC}).\par
\begin{figure}[h]
	\centering
	\includegraphics[width=120mm]{./pdf/LHC.pdf}
	\caption{LHC加速器の外観\cite{cern}.}
	\label{fig:LHC}
\end{figure}
CERNの加速器のビームは,小さな加速器から大きな加速器へと次々にエネルギーを上げながらLHCまで運ばれる(図\ref{fig:LHCs}).LHCでは,ビームは超高真空に保たれた2本のビームパイプ内を逆方向に進む.超電導磁石によって作られた高磁場が,7~TeVの陽子ビームを保持し,ビームを収束させることで,衝突点で素粒子反応を発生する.\par
\begin{figure}[h]
	\centering
	\includegraphics[width=150mm]{./pdf/LHCs.pdf}
	\caption{CERNの加速器群とLHC\cite{cern}.}
	\label{fig:LHCs}
\end{figure}
LHCには4つの大きな実験グループがあり,衝突点にはそれぞれの目的に合わせたALICE (A Large Ion Collider Experiment),ATLAS (A Toroidal LHC ApparatuS),CMS (Compact Muon Solenoid),LHCb (LHC beauty)と呼ばれる検出器を設置している.ALICEは鉛の原子核-原子核衝突によって高温状態を作り出すことで,粒子がクォーク・グルーオンプラズマからどのように生成されたかを研究する.ATLASとCMSは,陽子-陽子衝突によって発生した粒子の解析を行うことで,質量の起源や暗黒物質・暗黒エネルギーなどの解明を目指している.LHCbはbクォークの精密測定を目的としているが,LHCには他にもより限られた目的を持つLHCf (LHC forward),
TOTEM (TOTal Elastic and diffractive cross-section measurement),MoEDAL (Monopole and Exotics Detector at LHC)と呼ばれる検出器もあり,LHCは様々な物理に対して今後も期待されている.\par
2026年に開始するLHCの高輝度化計画(HL-LHC計画)では,入射器や収束磁石の改良により,瞬間輝度が現在の5倍以上に増強される予定である.これに伴い,検出器が置かれる放射線環境はより厳しくなるため,実現に向けて多くの関連分野で開発が進められている.
%----------------------------------------------------------------------%
\section{ATLAS}
ATLAS (A Toroidal LHC ApparatuS)は,LHCの衝突点のひとつに設置された汎用型粒子検出器である.全長44~m,高さ25~m,重量7000~tの大型検出器であるが,複数の検出器が組み合わされている(図\ref{fig:ATLAS}).内側から内部飛跡検出器,電磁カロリメータ,ハドロンカロリメータ,ミュー粒子検出器が設置されている.超電導ソレノイド磁石によって作られる2~Tの磁場中に置かれた内部飛跡検出器は,磁場で曲げられた荷電粒子の飛跡再構成や運動量測定を行う.電磁カロリメータは,電子・光子のエネルギー測定や電磁シャワーの位置測定を行う.ハドロンカロリメータはジェットのエネルギー測定を行い,トロイダル磁石によって作られる磁場中に置かれたミュー粒子検出器はミュー粒子の運動量測定を行う.
\begin{figure}[H]
	\centering
	\includegraphics[width=120mm]{./pdf/ATLAS.pdf}
	\caption{ATLAS\cite{cern}.}
	\label{fig:ATLAS}
\end{figure}
%----------------------------------------------------------------------%
\subsection*{内部飛跡検出器}
内部飛跡検出器は,全長5.3~m,直径2.1~mであり,ATLASの最も内側に位置する.超電導ソレノイド磁石が作る
2~Tの磁場によって曲げられた荷電粒子の曲率半径を求めることで,粒子の運動量測定と飛跡再構成による崩壊点測定を行う.現在のATLASでは内側からピクセル検出器,マイクロストリップ検出器,ストローチューブ型遷移輻射検出器という構造がとられている(図\ref{fig:ITK}).しかしHL-LHC計画では,ガス検出器であるストローチューブ型遷移輻射検出器は廃止され,全ての検出器が新しいシリコン検出器に置き換わることが決定している.これに向けて,放射線耐性に優れたセンサーの開発が進められている.
\begin{figure}[H]
	\centering
	\includegraphics[width=120mm]{./pdf/ITK.pdf}
	\caption{内部飛跡検出器\cite{cern}.}
	\label{fig:ITK}
\end{figure}
%----------------------------------------------------------------------%








	% 本文1
%%----------------------------------------------------------------------%
\chapter{半導体検出器}
\label{chap:semicon}
%----------------------------------------------------------------------%
\section{動作原理}
%----------------------------------------------------------------------%
\subsection{半導体}
通常原子が単独で存在する場合と,結晶中のように複数存在する場合では,電子のエネルギー状態が異なる.シリコンでは,原子同士が接近すると,$3s$, $3p$準位は分裂してエネルギー帯を形成する (図\ref{fig:Si}).電子の占有が許される領域を許容帯(allowed band)と呼ばれ,低い方のエネルギー帯である価電子帯(valence band)と,高い方のエネルギー帯である伝導帯(conduction band)に分かれる.一方これらの間に挟まれるエネルギー領域は,電子の準位が存在しない禁制帯(forbidden band)であり,このエネルギー幅を禁制帯幅(energy gap,$E_{g}$)と呼ぶ.\par
\begin{figure}[h]
	\centering
	\includegraphics[width=120mm]{./pdf/Si.pdf}
	\caption{シリコンのエネルギー帯構造.}
	\label{fig:Si}
\end{figure}
物質の導電性はエネルギー帯構造によって説明することができる.絶縁体は禁制帯幅が大きく,室温では価電子帯から伝導帯に電子が移ることはできないので,電流は流れにくい.一方導体は,価電子帯と伝導帯のエネルギー帯が重なっているため,電子は自由に移動することができ,電流は流れやすくなる.半導体の場合は,価電子帯と伝導帯が離れてはいるが,絶縁体に比べて禁制帯幅が小さいので,室温における原子の熱振動による熱エネルギーでも,電子が伝導帯に移ることができる (図\ref{fig:InSeCo}).\par
\begin{figure}[h]
	\centering
	\includegraphics[width=120mm]{./pdf/InSeCo.pdf}
	\caption{絶縁体・半導体・導体のエネルギー帯.}
	\label{fig:InSeCo}
\end{figure}
%----------------------------------------------------------------------%
\subsection{半導体の種類}
不純物のない半導体は真性半導体(intrinsic semiconductor)と呼ばれる.真性半導体にキャリアとして存在するのは,熱励起によって価電子帯から伝導帯に移った電子と,価電子帯に残された正孔(hole)のみである.そのため伝導体にある電子の数と,価電子帯にある正孔の数は必ず一致する (図\ref{fig:Thermal}).\par
\begin{figure}[h]
	\centering
	\includegraphics[width=100mm]{./pdf/Thermal.pdf}
	\caption{熱励起.}
	\label{fig:Thermal}
\end{figure}
真性半導体はキャリア密度が温度変化に対して敏感に変化するため,密度を制御することが難しく,デバイスに利用されることは少ない.このため実用的には,真性半導体にある種類の不純物元素を添加することにより,キャリアの一方のみを増加させるなど,半導体の抵抗率を自由に制御する手法が用いられる.このような不純物を含んだ半導体を不純物半導体(extrinsic semiconductor)という.不純物半導体の中で,密度の高い方のキャリアを多数キャリア,低い方のキャリアを少数キャリアと呼び,多数キャリアが電子である半導体をN型半導体,多数キャリアが正孔である半導体をP型半導体という.\par

\subsubsection*{N型半導体}
シリコン結晶に5族の不純物元素(P, As等)を添加すると,格子点上のシリコン原子と置換することで結晶格子の一部となる.5価の原子が持つ5個の価電子の内,4個は隣接するシリコン原子との共有結合に用いられるが,残りの1個は結合には関与せず,5価の原子にゆるく結合されている.新たに作られたこの準位はドナー準位(donor level, $E_{D}$)と呼ばれ,比較的小さなイオン化エネルギー($E_{C}-E_{D}$)のみで電子は伝導帯に移ることができる (図\ref{fig:Ntype}). 室温ではほとんどの電子が伝導帯に移るため,1個の5族元素は,1個の電子キャリアを結晶に与えることになる.そのためこの5族元素をドナーという.ドナーが添加された半導体はキャリアが負電荷(negative charge)になることから,N型半導体と呼ばれている.
\begin{figure}[h]
	\centering
	\includegraphics[width=150mm]{./pdf/Ntype.pdf}
	\caption{N型半導体の結晶構造とエネルギー帯.}
	\label{fig:Ntype}
\end{figure}

\subsubsection*{P型半導体}
シリコン結晶に3族の不純物元素(B等)を添加すると,格子上のシリコン原子と置換することで結晶格子の一部となる.3価の原子が持つ3個の価電子は,隣接するシリコン原子3個との共有結合に用いられる.さらにもう1個の共有結合のために周囲の価電子を1個受け取る.そのためこの3族元素をアクセプタという.受け取る際に,結晶中に1個の正孔を放出する.新たに作られたこの準位はアクセプタ準位と呼ばれ,室温ではほとんどのアクセプタがイオン化し,価電子帯に正孔を放出する.アクセプタが添加された半導体はキャリアが正電荷(positive charge)になることから,P型半導体と呼ばれている.
\begin{figure}[h]
	\centering
	\includegraphics[width=150mm]{./pdf/Ptype.pdf}
	\caption{P型半導体の結晶構造とエネルギー帯.}
	\label{fig:Ptype}
\end{figure}
%----------------------------------------------------------------------%
\subsection{PN接合}
\label{sub:pn}
P型半導体とN型半導体が接合すると,接合部を境にキャリアに大きな密度差が生じる.N型半導体の電子はP型領域に拡散し,正孔と再結合して消滅する.逆にP型領域の正孔はN型領域へと拡散し,電子と再結合する.一方,ドナーイオンやアクセプタイオンは格子上から動くことができないため,P型領域は負,N型領域は正の空間電荷が生じる.この領域は空乏層(Depletion Region)と呼ばれ,キャリアはほとんど存在しない (図\ref{fig:NoBias}).\par
\begin{figure}[h]
	\centering
	\includegraphics[width=95mm]{./pdf/NoBias.pdf}
	\includegraphics[width=90mm]{./pdf/name.pdf}
	\caption{PN接合のエネルギー帯.}
	\label{fig:NoBias}
\end{figure}
この空間電荷による電場は,拡散電位という電位差$V_{d}$を生じさせる.これと準位の関係は$E_{CP}-E_{CN}=eV_{d}$で書ける.また,電場の向きは,N型領域からP型領域であり,この電場によるドリフトの向きは,キャリア密度差による拡散の向きと逆向きになる.この2つの働きが釣り合うのが, PN接合の熱平衡状態である.\par

\subsubsection*{順方向バイアス}
P型領域側が正,N型領域側が負となるように電圧$V_{F}$を加えると,その分,P型領域に対するN型領域の電子のエネルギーは相対的に高くなる (図\ref{fig:FoBias}).つまり障壁の高さが低くなる($E_{CP}-E_{CN}=e(V_{d}-V_{F})$)ので,P型領域の多数キャリアである正孔は障壁を越えてN型領域へ,N型領域の多数キャリアである電子も障壁を越えてP型領域へ拡散により移動でき,電流が流れるようになる.この方向に加える外部電圧を順方向バイアスという.\par
\begin{figure}[h]
	\centering
	\includegraphics[width=84mm]{./pdf/FoBias.pdf}
	\caption{PN接合の順方向バイアス印加時のエネルギー帯.}
	\label{fig:FoBias}
\end{figure}

\subsubsection*{逆方向バイアス}
P型領域側が負,N型領域側が正となるように電圧$V_{R}$を加えると,その分,P型領域に対するN型領域の電子のエネルギーは相対的に低くなる (図\ref{fig:ReBias}).つまり障壁の高さが高くなる($E_{CP}-E_{CN}=e(V_{d}+V_{R})$)ので,多数キャリアは他方の領域へ拡散により移動することができない.この方向に加える外部電圧を逆方向バイアスという.空乏層端の少数キャリアは,電位障壁を降り他方に移動できるが,その密度は逆方向バイアスに依存せず一定なので,ほとんど電流は流れない.\par
\begin{figure}[h]
	\centering
	\includegraphics[width=84mm]{./pdf/ReBias.pdf}
	\caption{PN接合の逆方向バイアス印加時のエネルギー帯.}
	\label{fig:ReBias}
\end{figure}
%----------------------------------------------------------------------%
\subsection{半導体検出器}
半導体検出器はPN接合を基礎とする.PN接合の空乏層に荷電粒子や十分なエネルギーを持つ光子が入射すると,そのエネルギーにより電子正孔対が生成される.この電子や正孔を信号として収集するのが半導体検出器である.空乏層は厚い方が生成される電子正孔対は増加するため, 逆方向バイアスで動作させる.逆方向バイアスは,電子と正孔を電極側により引き寄せるため,信号を効率的に収集するという意味においても適している.図\ref{fig:Detector}に典型的なストリップ型半導体検出器の構造を示す.大きく分けてバルクにN型を用いて電極をP$^{+}$にするP$^{+}$-in-N型センサーと,バルクにP型を用いて電極をN$^{+}$にするN$^{+}$-in-P型センサーがあるが,現在HL-LHC計画に向けて開発されているのは,N$^{+}$-in-P型センサーである.\par
\begin{figure}[h]
	\centering
	\includegraphics[width=65mm]{./pdf/DetectorN.pdf}
	\includegraphics[width=65mm]{./pdf/DetectorP.pdf}
	\caption{典型的な半導体検出器の構造. N型バルク(左)とP型バルク(右).}
	\label{fig:Detector}
\end{figure}
%----------------------------------------------------------------------%
\section{半導体検出器の特性}
%----------------------------------------------------------------------%
\subsection{IV特性}
\label{sub:iviv}
PN接合に順方向バイアス$V_{F}$を加えたときに流れる電流密度$J$は,次式で表される.
\begin{eqnarray*}
J=e(\frac{D_{p}p_{n0}}{L_{p}}+\frac{D_{n}n_{p0}}{L_{n}})({\rm exp}(\frac{eV_{F}}{k_{B}T})-1)
\end{eqnarray*}
\begin{eqnarray*}
D_{n,p}=電子あるいは正孔の拡散定数,L_{n,p}=電子あるいは正孔の拡散距離,\\
n_{p0}=P型領域の電子密度,p_{n0}=N型領域の正孔密度,k_{B}=ボルツマン定数
\end{eqnarray*}
$V_{F}$が大きくなると指数項の寄与が大きくなるため,順方向バイアスにおける電流は指数関数的に増加する.逆方向バイアスに対しても$J$は同様に表されるが,$V_{R}$が大きくなると指数項が1に対して無視できる程度に小さくなるため,逆方向バイアスでは小さい一定の電流が流れるようになる.半導体検出器は逆方向バイアスで動作するため,センサーにはほぼ一定の小さな電流が流れる.しかし,逆方向バイアスをさらに大きくしていくと,ある電圧で急激に大量の電流が流れる.このときの電圧を降伏電圧(breakdown voltage)と呼ぶ.以上よりIV特性の概略は図\ref{fig:IVbreak}のようになる.\par
\begin{figure}[H]
	\centering
	\includegraphics[width=75mm]{./pdf/breakdown.pdf}
	\caption{PN接合のIV特性概略図.}
	\label{fig:IVbreak}
\end{figure}
降伏電圧による大電流を流し続けるとセンサーを熱的に破壊してしまうため,印加電圧はこの降伏電圧より小さい必要がある.降伏の機構にはツェナー降伏(Zener breakdown)と雪崩降伏(avalanche breakdown,アバランシェ降伏)がある.ツェナー降伏は逆方向バイアスによってPN接合における価電子帯と伝導帯がの空間的な距離が縮まり,量子力学的なトンネル効果によって,価電子帯の電子が伝導帯へ直接通り抜ける現象である(図\ref{fig:break}左)).雪崩降伏は,空乏層の電場によって加速されたキャリアが,結晶格子上の原子内の価電子に衝突することで電子正孔対を生成し,さらにその生成された電子が同じことを繰り返していくことで起こる(図\ref{fig:break}(右)).一般に温度が高くなると禁制帯幅$E_{g}$が小さくなるため,トンネル現象が生じやすく,ツェナー降伏はより低電圧で起こるようになる.一方,温度の増大に伴い格子振動が激しくなるため,キャリアの移動度は小さくなり,アバランシェ降伏はより高電圧で起こるようになる.
\begin{figure}[H]
	\centering
	\includegraphics[width=100mm]{./pdf/break.pdf}
	\caption{ツェナー降伏(左)と雪崩降伏(右).}
	\label{fig:break}
\end{figure}
%----------------------------------------------------------------------%
\subsection{CV特性}
\label{sub:cv}
PN接合における空間電荷の分布と電場・電位を図\ref{fig:cvcv}に示す.ただし,接合面を$x=0$とし,$x$が負の領域をアクセプタ密度が一定のP型領域,$x$が正の領域をドナー密度が一定のN型領域であると仮定した.さらに,空乏層内には不純物イオンによる空間電荷が存在し,空乏層外では不純物イオンとキャリアによって電気的に中性であるという空乏近似を用いると,空乏層厚$w$は次式で表すことができる.
\begin{eqnarray*}
w=x_{p}+x_{n}=\sqrt{\frac{2\epsilon_{0}\epsilon_{Si}(N_{a}+N_{d})}{eN_{a}N_{d}}(V_{d}+V_{R})}
\end{eqnarray*}
\begin{eqnarray*}
N_{a,d}=アクセプターあるいはドナー密度,\epsilon_{0}=真空の誘電率,\epsilon_{Si}=シリコンの比誘電率
\end{eqnarray*}
N$^{+}$-in-P型センサーにおける電極とバルク部の空乏化では,$N_{a}\gg N_{d}$であり,式からバルク部の$N_{d}$が小さいほど同じ電圧での空房層厚は厚くなることが分かる.
空乏層の単位面積当たりの静電容量は,次式で表される.
\begin{eqnarray*}
C=\frac{\epsilon_{0}\epsilon_{Si}}{w}=\sqrt{\frac{\epsilon_{0}\epsilon_{Si}eN_{a}N_{d}}{2(N_{a}+N_{d})}\frac{1}{(V_{d}+V_{R})}}
\end{eqnarray*}
これより,CV特性の概略は図\ref{fig:cvcvcv}のようになる.また,このCV特性の傾きは,
\begin{eqnarray*}
\frac{dC}{dV_{R}}=-\frac{1}{2}\frac{C}{V_{d}+V_{R}}
\end{eqnarray*}
であるので,N$^{+}$-in-P型センサーにおける電極とバルク部の空乏化では,バルク部の$N_{d}$が小さいほど静電容量の電圧に対する変化率は大きくなる.
\begin{figure}[h]
\begin{minipage}[b]{0.6\hsize}
	\centering
	\includegraphics[width=100mm]{./pdf/cvcv.pdf}
	\caption{PN接合の空間電荷分布と電場・電位.}
	\label{fig:cvcv}
\end{minipage}
\begin{minipage}[b]{0.4\hsize}
	\centering
	\includegraphics[width=60mm]{./pdf/capa.pdf}
	\caption{PN接合のCV特性概略図.}
	\label{fig:cvcvcv}
\end{minipage}
\end{figure}
%----------------------------------------------------------------------%
\subsection{放射線損傷}
\label{sub:damage}
半導体検出器の放射線損傷は,大きく表面損傷とバルク損傷に分けられる(図\ref{fig:damageS}).
\subsubsection*{表面損傷}
放射線が照射されると,バルク部の空乏層だけでなく,SiO$_{2}$層においても電子正孔対が生成される.電子と正孔では移動度が異なるため,移動度の小さい正孔のみがSiO$_{2}$層にトラップされてしまう.これが蓄積していくことで,SiO$_{2}$層は正に帯電する.このSiO$_{2}$層の電荷は,センサーの性能に影響を与え,暗電流を増加させたり,表面電場の上昇による降伏電圧の変化を引き起こす.この損傷はバルク部では起こらず,酸化膜のある表面でのみ起こるため表面損傷と呼ばれる.蓄積する電荷量には限界があるので,一般的に表面損傷は線量増大に対して飽和する傾向がある.
\subsubsection*{バルク損傷}
センサーに放射線が入射すると,電子の励起や,結晶格子上の原子との衝突などによりエネルギーを失う.ここで,衝突により原子が結晶表面まで弾き出され,原子空孔(vacancy)のみが残されたものをショットキー欠陥と呼び,結晶表面までは移動せずに結晶格子間で止まった場合,その格子間原子と原子空孔の対をフレンケル欠陥と呼ぶ.これらの格子欠陥は,結晶に新たな不純物準位を形成するため,電子・正孔がトラップされることによる収集電荷量の低下や,暗電流の増加を招く.これら格子欠陥はP型やN型など種々に振舞うが,シリコンの場合は統合的にはP型の不純物の生成の方が多い.熱中性子が衝突すると,周囲の格子が集団で破壊されることも多い.そのため,中性子による損傷が陽子による損傷よりも大きいことが分かっている.
\begin{figure}[h]
	\centering
	\includegraphics[width=100mm]{./pdf/damageS.pdf}
	\caption{表面損傷(左)とバルク損傷(右).}
	\label{fig:damageS}
\end{figure}









	% 本文2
%%----------------------------------------------------------------------%
\chapter{LGADの性能評価}
\label{chap:LGAD}
%----------------------------------------------------------------------%
\section{背景}
現行のP$^{+}$-in-N型ピクセルセンサーに対して,現在HL-LHC計画に向けて開発が進められているのは,N$^{+}$-in-P型ピクセルセンサーである.第\ref{sub:damage}項で前述したように,バルク損傷はP型不純物として振る舞うために,高放射線環境下ではP$^{+}$-in-N型センサーのN型バルクがP型に型反転し,全空乏化しなければ信号が読み出せなくなる.一方,現在開発が進められているN$^{+}$-in-P型センサーでは,P型バルクなので型反転を起こさず,常に部分空乏化でも信号の読み出しが可能となる.このN$^{+}$-in-P型センサーの内部に増幅機能を持たせた検出器が,本研究で研究開発を行なったLGAD (Low Gain Avalanche Detector)である.信号増幅機能を持つことで,従来型($\sim150~\mu{\rm m}$)よりもさらに薄いセンサー($\sim50~\mu{\rm m}$)で信号読み出しに十分な信号を得ることができる.センサーが薄くなることで,信号の立ち上がりが速くなり,時間分解能は向上する.また,物質量が抑えられるために,センサー内での散乱が少なく,低エネルギー粒子の運動量測定精度が向上する.この特徴はMPPC (Multi-Pixel Photon Counter)も持つが,MPPCはガイガーモードで動作するため,入射粒子数によって信号量は変化しない.一方,LGADではリニアモードで動作するために,入射粒子数をパルスの変化によって検出することができる点で新しい.この増幅機能は,高増幅ではノイズも増幅してしまうため,高エネルギー実験に適したS/N比を実現するためには10倍程度の低増幅が望ましいと言われている.また,LGADは時間分解能が良いため,数十psの時間分解能があれば性能が飛躍的に改善するPET(Positron Emission Tomography)装置など,医療機器への応用に対しても期待されている.\par
%----------------------------------------------------------------------%
\subsection{LGADの構造}
LGADの基本的な構造を図\ref{fig:LGADbase}に示す.従来のN$^{+}$-in-P型センサー同様,P型バルクにN$^{+}$電極をインプラントした検出器であるが,LGAD特有の構造は,N$^{+}$電極直下あるP-well層である.このP-well層は高抵抗のP型バルクよりも不純物濃度が高く,N$^{+}$電極とのPN接合により高電場を形成する.この高電場により,電極まで移動する電子がアバランシェを起こし,信号が増幅される.以降,このP-well層を増幅層と呼ぶ.\par
\begin{figure}[h]
	\centering
	\includegraphics[width=120mm]{./pdf/LGAD.pdf}
	\caption{LGADの基本構造.}
	\label{fig:LGADbase}
\end{figure}
%----------------------------------------------------------------------%
\subsection{サンプル}
本研究で使用したセンサーは浜松ホトニクス社で製造されたもので,大きくピクセル型センサー(図\ref{fig:LGAD}(左))とストリップ型センサー(図\ref{fig:LGAD}(右))に分けられる.増幅層の不純物濃度は4段階あり,Aが最も少なく,B,C,Dの順で多くなっている.空乏化できるバルク部の厚さ(活性層厚,Active Thickness)にも50~$\mu\rm{m}$と80~$\mu\rm{m}$の2種類がある.\par
\begin{figure}[H]
\begin{minipage}{0.48\hsize}
	\raggedleft
	\includegraphics[width=50mm]{./pdf/LGADpixel.pdf}
 \end{minipage}
 \begin{minipage}{0.52\hsize}
	\includegraphics[width=50mm]{./pdf/LGADstrip.pdf}
 \end{minipage}
 	\caption{ピクセル型センサー(左)とストリップ型センサー(右).}
	\label{fig:LGAD}
\end{figure}
%----------------------------------------------------------------------%
\subsubsection*{ピクセル型LGAD}
本研究で使用したピクセル型センサーは,大きさ2.5~mm$\times$2.5~mm,厚さ150~$\mu\rm{m}$であり,1~mm$\phi$の受光面を持つ.本研究で用いたサンプル名を表\ref{tab:sampleP}にまとめた.
\begin{table}[h]
	\centering
	\caption{ピクセル型LGADのサンプル名}
	\vspace{5truemm}
	\begin{tabular}{@{\hspace{0.5cm}}c@{\hspace{1cm}}c@{\hspace{1cm}}c@{\hspace{1cm}}c@{\hspace{1cm}}c@{\hspace{0.5cm}}}\hline
	\textbf{Active Thickness}& \textbf{Dose A}& \textbf{Dose B}& \textbf{Dose C}& \textbf{Dose D}\\
	\hline\hline
	50~$\mu\rm{m}$& 50A& 50B& 50C& 50D\\
	\hline
	80~$\mu\rm{m}$& 80A& 80B& 80C& 80D\\
	\hline
	\end{tabular}
	\label{tab:sampleP}
\end{table}
%----------------------------------------------------------------------%
\subsubsection*{ストリップ型LGAD}
本研究で使用したストリップ型センサーは,大きさ6~mm$\times$12~mm, 厚さ150~$\mu\rm{m}$であり,80~$\mu\rm{m}$ピッチで50ストリップある.本研究で用いたサンプル名を表\ref{tab:sampleS}にまとめた.
\begin{table}[h]
	\centering
	\caption{ストリップ型LGADのサンプル名}
	\vspace{5truemm}
	\begin{tabular}{@{\hspace{0.5cm}}c@{\hspace{0.7cm}}r@{\hspace{0.7cm}}r@{\hspace{0.7cm}}r@{\hspace{0.7cm}}r@{\hspace{0.7cm}}c@{\hspace{0.5cm}}}\hline
	\textbf{Active Thickness}& \textbf{Dose A}& \textbf{Dose B}& \textbf{Dose C}& \textbf{Dose D}\\%&  $\ast$\\
	\hline\hline
%	50~$\mu\rm{m}$& S50A& S50B& S50C& S50D& $\times$\\
%	\hline
%	80~$\mu\rm{m}$& S80A& S80B& S80C& S80D& $\times$\\
%	\hline
	50~$\mu\rm{m}$& GB50A& GB50B& GB50C& GB50D\\%& $\circ$\\
	\hline
	80~$\mu\rm{m}$& GB80A& GB80B& GB80C& GB80D\\%& $\circ$\\
	\hline
%	\multicolumn{6}{r}{$\ast$ Higher-Breakdown-Voltage-Design}
	\end{tabular}
	\label{tab:sampleS}
\end{table}
%----------------------------------------------------------------------%
\subsubsection*{P型バルク}
P型バルク厚は150~$\mu$mであるが,活性層を除いて,P$^{+}$の不活性層を拡散侵入させている.また,Active Thicknessが50~$\mu$mと80~$\mu$mのサンプルでは,活性層の比抵抗が異なる.
%異なり,それぞれ3$\sim$8~k$\Omega$cmと1~k$\Omega$cmである.
%----------------------------------------------------------------------%
\section{IV測定}
%----------------------------------------------------------------------%
\subsection{測定方法}
\label{sub:IVsetup}
N$^{+}$電極とP$^{+}$電極に逆バイアス電圧を加えたとき,ピクセル型センサーに流れる電流値を測定した(図\ref{fig:IV}(左)).LED光はファンクションジェネレータから1~kHzの矩形パルス(Duty cycle$=50~\%$)を出力することで点灯させ,センサーとの距離($\sim$4~cm)は一定に保たれるよう固定した(図\ref{fig:IV}(中央, 右)).
%\begin{figure}[H]
%	\centering
%	\includegraphics[width=100mm]{./pdf/IVsetup.pdf}
%	\caption{IV測定のセットアップ.}
%	\label{fig:IVsetup}
%\end{figure}
%\begin{figure}[H]
%\begin{minipage}{0.48\hsize}
%	\raggedleft
%	\includegraphics[width=60mm]{./pdf/IVtate.pdf}
%\end{minipage}
%\begin{minipage}{0.52\hsize}
%	\includegraphics[width=60mm]{./pdf/IVyoko.pdf}
%\end{minipage}
%	\caption{IV測定時のセンサーとLED光の位置関係.}
%	\label{fig:IVlight}
%\end{figure}
\begin{figure}[H]
\begin{minipage}{0.36\hsize}
	\includegraphics[width=57mm]{./pdf/IVsetup.pdf}
\end{minipage}
\begin{minipage}{0.3\hsize}
	\raggedleft
	\includegraphics[width=57mm]{./pdf/IVtate.pdf}
\end{minipage}
\begin{minipage}{0.3\hsize}
	\raggedright
	\includegraphics[width=57mm]{./pdf/IVyoko.pdf}
\end{minipage}
	\caption{IV測定のセットアップ(左). センサーとLED光の位置関係(中央, 右).}
	\label{fig:IV}
\end{figure}
%----------------------------------------------------------------------%
\subsection{ピクセル型センサーの耐電圧}
\label{sub:weak}
ピクセル型センサーのIV特性を調べる際,いくつかのセンサーがスパークした.スパークは2種類に分けられる.グランド側のアルミとN$^{+}$電極に繋げたワイヤーボンドが近いために,ワイヤーボンドの下側にあるアルミが焦げてしまったもの(図\ref{fig:spark}(左))と,リング状のP$^{+}$電極とN$^{+}$電極の間が十分広くないためにスパークしたもの(図\ref{fig:spark}(右))である.ワイヤーボンドによるスパークは,アルミとの距離を十分にとって繋ぐことで防止できた.一方,リング上のスパークはピクセル型センサーの設計上の欠陥であるため,高電圧($\textgreater$800~V)をかける際にはスパークを防げなかった.以降の測定結果に欠けているサンプルがあるのは,これらのスパークのためである.
%\begin{figure}[H]
%\begin{minipage}{0.5\hsize}
%	\raggedleft
%	\includegraphics[width=65mm]{./pdf/spark1.pdf}
%\end{minipage}
%\begin{minipage}{0.5\hsize}
%	\raggedright
%	\includegraphics[width=65mm]{./pdf/spark2.pdf}
%\end{minipage}
%	\caption{ピクセル型センサーのスパーク(アルミの焦げ:左, リング上:右).}
%	\label{fig:spark}
%\end{figure}
\begin{figure}[H]
	\centering
	\includegraphics[width=60mm]{./pdf/spark1.pdf}
	\includegraphics[width=60mm]{./pdf/spark2.pdf}
	\caption{ピクセル型センサーのスパーク(アルミの焦げ:左, リング上:右).}
	\label{fig:spark}
\end{figure}
%----------------------------------------------------------------------%
\subsection{暗電流との比較}
\label{sub:leak}
LED光応答による信号を評価するため,暗電流(LED光を当てていないときの電流値)の測定を行なった.以降の測定では,この暗電流に対して2桁程度電流値が大きくなるように,LED光の強度を調整した.暗電流とLED赤外光($\lambda\sim$850~nm)応答の結果をまとめて図\ref{fig:leak}に示す.いずれも20~℃での測定結果である.\par
\begin{figure}[H]
\begin{minipage}{0.5\hsize}
	\centering
	\includegraphics[width=75mm]{./graph/50.pdf}
\end{minipage}
\begin{minipage}{0.5\hsize}
	\centering
	\includegraphics[width=75mm]{./graph/80.pdf}
\end{minipage}
 	\caption{暗電流とLED赤外光応答のIV特性比較.}
	\label{fig:leak}
\end{figure}
暗電流は,全てのサンプルについて同程度であるのに対して,LED赤外光応答は暗電流とは異なる電流値の上昇が見え,その上昇の仕方はサンプル毎に異なる.これはLGADの増幅機能による効果だと考えられる.増幅層の不純物濃度は高く,Active Thicknessは薄い(50~$\mu$m)ほど,より低電圧でも増幅が確認できる.また,暗電流はある一定の電圧で急激に上昇する.この変化について,サンプル80Dで詳しく調べるため,暗電流が急上昇する500~V付近では1~V間隔で測定した結果が図\ref{fig:iv}である.500~V付近でLED光応答の電流値に比べて暗電流が急激に上昇していることがわかる.S/N比の変化を評価するため,LED光応答(Current)と暗電流(Leakage)に対して,信号(Signal)とノイズ(Noise)を次のように定義する.
\[Signal=Current-Leakage, Noise=\sqrt{Leakage}\]
%\[Signal/Noise=(Current-Leakage)/\sqrt{Leakage}\]
これによって求まるSignal/Noiseの絶対値は,S/N比ではないが,S/N比に比例する量である.サンプル80DにおけるSignal/Noiseの値を図\ref{fig:iv}から求めた結果を図\ref{fig:sn}に示す.
\begin{figure}[H]
\begin{minipage}{0.5\hsize}
	\centering
	\includegraphics[width=75mm]{./graph/80Dxxx.pdf}
	 	\caption{Pixel 80DのIV特性.}
	\label{fig:iv}
\end{minipage}
\begin{minipage}{0.5\hsize}
	\centering
	\includegraphics[width=75mm]{./graph/80Dooo.pdf}
	 	\caption{Pixel 80Dの電圧に伴うS/Nの変化.}
	\label{fig:sn}
\end{minipage}
\end{figure}
S/N比は信号の増加に伴い上昇し,$-$498~Vで最大になった後に急激に下降する.つまり,S/N比は暗電流が急激に上昇する直前に最大となることが分かる.各サンプルについてS/N比が下降する前の増幅率を図\ref{fig:leak}から求め,表\ref{tab:sn}にまとめた.ただし,増幅率は$-$100~Vにおける電流値を基準とした.これより,良いS/N比には100倍を超えない低増幅が適していることが確かめられた.また,増幅層の不純物濃度が高く, Active Thicknessが薄いほど,低電圧でS/N比が最適になると言える.\par
\begin{table}[H]
	\centering
	\caption{S/N比が最大となる電圧値と増幅率}
	\vspace{5truemm}
	\begin{tabular}{@{\hspace{0.5cm}}c@{\hspace{0.7cm}}c@{\hspace{0.7cm}}r@{\hspace{0.7cm}}}\hline
	\textbf{Sample Name}& \textbf{Voltage[V]}& \textbf{Gain}\\
	\hline\hline
	50C& 470& 31.9\\
	\hline
	50D& 340& 27.2\\
	\hline
	80A& 680& 9.4\\
	\hline
	80B& 680& 14.6\\
	\hline
	80C& 620& 13.8\\
	\hline
	80D& 490& 29.0\\
	\hline
	\end{tabular}
	\label{tab:sn}
\end{table}
%----------------------------------------------------------------------%
\newpage
\subsection{温度依存性}
\label{sub:temp}
IV特性の温度依存性を見るため,LED緑色光($\lambda\sim$565~nm)応答を$-$20$\sim$60~℃について20~℃間隔で測定した.Active Thickness 80~$\mu$mのサンプルにおける測定結果を図\ref{fig:tdep}に示す(Active Thickness 50~$\mu$mのサンプルについては付録\ref{T}参照).いずれのサンプルにおいても,温度が低下するほど電流値は大きくなっている.第\ref{sub:iviv}節で前述したように,電流の上昇がツェナー降伏によるものであると,温度が低下するほど電流値は小さくなるはずである.つまりこの結果は,電流の上昇が増幅層におけるアバランシェによるものであることを示している.
\begin{figure}[H]
\begin{minipage}{0.5\hsize}
	\centering
	\includegraphics[width=75mm]{./graph/Tdep80A.pdf}
\end{minipage}
\begin{minipage}{0.5\hsize}
	\centering
	\includegraphics[width=75mm]{./graph/Tdep80B.pdf}
\end{minipage}
\begin{minipage}{0.5\hsize}
	\centering
	\includegraphics[width=75mm]{./graph/Tdep80C.pdf}
\end{minipage}
\begin{minipage}{0.5\hsize}
	\centering
	\includegraphics[width=75mm]{./graph/Tdep80D.pdf}
\end{minipage}
 	\caption{IV特性の温度依存性(Active Thickness 80~$\mu$m).緑色LEDに対する応答.}
	\label{fig:tdep}
\end{figure}
%----------------------------------------------------------------------%
\newpage
\subsection{波長依存性}
\label{sub:wave}
LGAD内部の構造を理解するために,IV特性のLED波長依存性を測定した.測定に用いたLEDは青色・緑色・赤色・赤外の4種類である。それぞれの波長とシリコンへの侵入長を表\ref{tab:lambda}にまとめた.ただし侵入長は図\ref{fig:D}から見積もった.Active Thickness 80~$\mu$mのサンプルにおける測定結果は図\ref{fig:WLdep}(左)に示す(Active Thickness 50~$\mu$mのサンプルについては付録\ref{W}参照)).\par
\begin{table}[H]
	\centering
	\caption{測定に用いたLED光の波長とシリコンへの侵入長}
	\vspace{5truemm}
	\begin{tabular}{@{\hspace{0.5cm}}c@{\hspace{0.7cm}}c@{\hspace{0.7cm}}c@{\hspace{0.5cm}}}\hline
	\textbf{LED}& \textbf{Wavelength}& \textbf{Absorption Depth}\\
	\hline\hline
	Blue& 464 nm& 0.5 $\mu$m\\
	\hline
	Green& 565 nm& 2.0 $\mu$m\\
	\hline
	Red& 627 nm& 3.0 $\mu$m\\
	\hline
	Infra-red& 850 nm& 20 $\mu$m\\
	\hline
	\end{tabular}
	\label{tab:lambda}
\end{table}
\begin{figure}[H]
	\centering
	\includegraphics[width=95mm]{./pdf/silicon.pdf}
	\caption{シリコンへの侵入長の波長依存性\cite{D}.}
	\label{fig:D}
\end{figure}
測定結果から波長が長いほど増幅が大きいが,降伏電圧は波長に依存しないことが分かった.また,0$\sim$100~Vに急激な電流値の上昇が見えたため,0$\sim$100~Vについて1~V間隔で測定し直したのが図\ref{fig:WLdep}(右)である.これを見ると,この上昇は増幅層の不純物濃度が高いほど高電圧まで続いていることが分かる.さらに増幅層の不純物濃度が大きく,波長が長いほど上昇が大きくなっている.しかし,緑色と赤色の上昇の差に比べて,赤色と赤外での上昇の差が小さいことから,増幅層は赤外の侵入長(20~$\mu$m)より浅く,赤色の侵入長(3.0~$\mu$m)より深いと推測できる.
\begin{figure}[H]
\begin{minipage}{0.55\hsize}
	\centering
	\includegraphics[width=75mm]{./graph/WLdep80Afull.pdf}
\end{minipage}
\begin{minipage}{0.45\hsize}
	\centering
	\includegraphics[width=75mm]{./graph/WLdep80A100.pdf}
\end{minipage}
\begin{minipage}{0.55\hsize}
	\centering
	\includegraphics[width=75mm]{./graph/WLdep80Bfull.pdf}
\end{minipage}
\begin{minipage}{0.45\hsize}
	\centering
	\includegraphics[width=75mm]{./graph/WLdep80B100.pdf}
\end{minipage}
\begin{minipage}{0.55\hsize}
	\centering
	\includegraphics[width=75mm]{./graph/WLdep80Cfull.pdf}
\end{minipage}
\begin{minipage}{0.45\hsize}
	\centering
	\includegraphics[width=75mm]{./graph/WLdep80C100.pdf}
\end{minipage}
\begin{minipage}{0.55\hsize}
	\centering
	\includegraphics[width=75mm]{./graph/WLdep80Dfull.pdf}
\end{minipage}
\begin{minipage}{0.45\hsize}
	\centering
	\includegraphics[width=75mm]{./graph/WLdep80D100.pdf}
\end{minipage}
 	\caption{IV特性のLED波長依存性(全体:左, 0$\sim$100 V:右).}
	\label{fig:WLdep}
\end{figure}
%----------------------------------------------------------------------%
\section{CV測定}
\label{sub:CV}
%----------------------------------------------------------------------%
N$^{+}$電極とP$^{+}$電極に逆バイアス電圧を加えたとき,空乏化により変化するストリップ型センサーの静電容量を測定した(図\ref{fig:CVsetup}). Active Thickness 80~$\mu$mのサンプルにおける測定結果を図\ref{fig:CV}(左)に示す(Active Thickness 50~$\mu$mのサンプルについては付録\ref{CV}参照).全空乏化が0$\sim$100~Vには終わることが分かったので,0$\sim$100~Vについて1~V間隔で測定し直した結果が図\ref{fig:CV}(右)である.LCRメータのテスト周波数は1500Hzに設定した.\par
\begin{figure}[H]
	\centering
	\includegraphics[width=120mm]{./pdf/CVsetup.pdf}
 	\caption{CV測定のセットアップ.}
	\label{fig:CVsetup}
\end{figure}
\begin{figure}[H]
\begin{minipage}{0.5\hsize}
	\centering
	\includegraphics[width=75mm]{./graph/60GB80CVfull.pdf}
\end{minipage}
\begin{minipage}{0.5\hsize}
	\centering
	\includegraphics[width=75mm]{./graph/60GB80CV100.pdf}
\end{minipage}
 	\caption{CV特性の増幅層不純物濃度依存性(全体:左, 0$\sim$100V:右).}
	\label{fig:CV}
\end{figure}
静電容量の変化から空乏化が段階的に起こっていることが分かる.この変化の様子から,空乏化がどの領域から起こるのかについて考察する.まず,Active Thickness 80~$\mu$mのサンプルが全空乏化したときの静電容量[F]を求める.センサーの大きさは6~mm$\times$12~mmであるが,周囲1~mm程度は空乏化できない領域なので,面積S=4~mm$\times$10~mm,厚さd=80~$\mu$mで考えれば良い.\par
\begin{eqnarray*}
C&=&\epsilon_{0}\epsilon_{Si}\frac{S}{d}\\
&=&(8.854\times10^{-12})\times12.0\times\frac{(4\times10^{-3)}\times(10\times10^{-3})}{80\times10^{-6}}~{\rm F}\\
&\simeq&53~{\rm pF}
\end{eqnarray*}
\begin{eqnarray*}
真空の誘電率\epsilon_{0}=8.854\times10^{-12}~{\rm F/m},シリコンの比誘電率\epsilon_{Si}=12.0
\end{eqnarray*}
同様に,増幅層が空乏化したときの静電容量も求める.ただし厚さは第\ref{sub:wave}節の考察より,6~$\mu$mとして計算する.
\begin{eqnarray*}
C&=&(8.854\times10^{-12})\times12.0\times\frac{(4\times10^{-3})\times(10\times10^{-3})}{6\times10^{-6}}~{\rm F}\\
&\simeq&708~{\rm pF}
\end{eqnarray*}
全空乏化後の静電容量は高電圧($\textgreater$60~V)での値に一致している.その後静電容量が低下していないことからも60~V付近で全空乏化していることが分かる.増幅層空乏化後の静電容量の計算値から,増幅層の空乏化は5$\sim$15~V付近で起こっていると考えられるが,図\ref{fig:CV}(右)の形からも,その前後の領域より傾きが小さく,不純物濃度の高い領域の空乏化が起こっていることが分かる.第\ref{sub:cv}節で前述したように,不純物濃度が高い半導体と低い半導体では,低い半導体の方が低電圧でも深くまで空乏化するため,CV特性の傾きは大きくなる.以上をまとめると,0$\sim$5~V付近でN$^{+}$電極の増幅層に接していないバルク部との境界から空乏化が始まり,次いでN$^{+}$電極と増幅層の境界,最後にバルク部全体との空乏化が起きていると考えることができる.
%----------------------------------------------------------------------%
%\newpage
\section{赤外レーザー測定}
\label{sub:Laser}
%----------------------------------------------------------------------%
ストリップ型センサーのNdYaGレーザー($\lambda=$1064 nm)応答をAlibavaを用いて測定した(図\ref{fig:Lasersetup}(左)).レーザー測定では狭い領域(1~$\mu$m程度)に限定して入射することができるため,ストリップ型センサーの増幅層がある領域(図\ref{fig:Lasersetup}(右)実線)とない領域(図\ref{fig:Lasersetup}(右)点線)とで,電荷収集量の電圧依存性に違いがあるかを調べた. 波長$\lambda=$1064~nmではセンサーのバルク部を貫通していると考えて良い(図\ref{fig:D}).結果を図\ref{fig:Laser}に示す.増幅層のある領域では,電圧が高くなるほど電荷収集量も増加しているが,増幅層のない領域では大きく変化していないことが分かる.つまり,IV特性で見えていた電流値の増加は,増幅層での信号の増幅であったことが明確となった.さらに,増加が始まる電圧を正確に知ることができなくても,この増幅層のある領域とない領域で電荷収集量の比をとれば,増幅率を評価することができる.図\ref{fig:Laser}の結果からは,増幅率が電圧によって図\ref{fig:gain}のように変化すると言える.この評価方法を用いて,他のサンプルについても測定を行うことで,さらなるLGADの構造理解へと繋げたい.\par
\begin{figure}[H]
\begin{minipage}[b]{0.6\hsize}
	\centering
	\includegraphics[width=85mm]{./pdf/Lasersetup.pdf}
\end{minipage}
\begin{minipage}[b]{0.3\hsize}
	\centering
	\includegraphics[width=40mm]{./pdf/Laser.pdf}
\end{minipage}
 	\caption{レーザー測定のセットアップ(左)とレーザー入射位置(右).}
	\label{fig:Lasersetup}
\end{figure}
\begin{figure}[H]
\begin{minipage}{0.5\hsize}
	\centering
	\includegraphics[width=75mm]{./graph/LaserN.pdf}
\end{minipage}
\begin{minipage}{0.5\hsize}
	\centering
	\includegraphics[width=75mm]{./graph/LaserS.pdf}
\end{minipage}
 	\caption{レーザー測定の結果.増幅層なし(左)と増幅層あり(右).}
	\label{fig:Laser}
\end{figure}
\begin{figure}[H]
	\centering
	\includegraphics[width=70mm]{./graph/gain.pdf}
 	\caption{増幅率の電圧依存性.}
	\label{fig:gain}
\end{figure}	% 本文3
%%----------------------------------------------------------------------%
\chapter{LGADの放射線耐性評価}
\label{chap:tolerance}
%----------------------------------------------------------------------%
\section{$\boldsymbol\gamma$線照射}
ピクセル型LGADの表面損傷を評価するため,2016年11月に国立研究開発法人量子科学技術研究開発機構の高崎量子応用研究所にて,$\gamma$線照射を行った.
 %----------------------------------------------------------------------%
\subsection{$^{60}$Co$\gamma$線照射環境と照射量}
高崎量子応用研究所は,コバルト第1棟・コバルト第2棟・食品照射棟の3つの$\gamma$線照射棟を有する国内最大の研究用照射施設である.今回使用したのは食品照射棟第2照射室で,$^{60}$Co線源からの$\gamma$線を遮蔽する厚さ1.3 mの重コンクリートの壁に囲まれている(図\ref{fig:Taka}(左)).$^{60}$Co線源がプールから照射室内へ上昇し,照射が行われる.$^{60}$Co線源は,天然物質の$^{59}$Coに原子炉で中性子を照射することで作られるが,$\beta$崩壊することでさらにNiに変化し,その際2つの$\gamma$線(1.173MeV, 1.333MeV)を放出する(図\ref{fig:Taka}(右)).\par
\begin{figure}[H]
\begin{minipage}{0.5\hsize}
	\centering
	\includegraphics[width=75mm]{./pdf/Ta.pdf}
 \end{minipage}
 \begin{minipage}{0.5\hsize}
	\centering
	\includegraphics[width=50mm]{./pdf/ka.pdf}
 \end{minipage}
 	\caption{食品照射棟\cite{gamma}(左)と$^{60}$Coの崩壊図(右).}
	\label{fig:Taka}
\end{figure}
測定当時の食品照射棟第2照射室における線量率分布は図\ref{fig:sa}に示した通りであり,これより照射量に応じて,照射レートと照射時間を表\ref{tab:ki}のように決定した.また表\ref{tab:takasam}には,照射を行ったサンプルを$\circ$で示し,サンプル数が十分になかったため,照射を行えなかったサンプルは$\times$,実験の過程でスパークしたサンプルは{\scriptsize$\triangle$}で示してある.
\begin{figure}[H]
	\centering
	\includegraphics[width=140mm]{./pdf/sa.pdf}
	\caption{食品照射棟第2照射室の線量分布.}
	\label{fig:sa}
\end{figure}
\begin{table}[H]
	\centering
	\caption{照射レートと照射時間}
	\vspace{5truemm}
	\begin{tabular}{@{\hspace{0.5cm}}c@{\hspace{1cm}}c@{\hspace{1cm}}c@{\hspace{0.5cm}}}\hline
	\textbf{Dose}& \textbf{Rate}& \textbf{Time}\\
	\hline\hline
	0.1MGy& 5kGy/h& 23h\\
	\hline
	1.0MGy& 2kGy/h& 500h\\
	\hline
	2.5MGy& 5kGy/h& 500h\\
	\hline
	\end{tabular}
	\label{tab:ki}
\end{table}
\begin{table}[H]
	\centering
	\caption{ピクセル型センサーの$\gamma$線照射サンプル一覧.}
	\vspace{5truemm}
	\begin{tabular}{|c||c|c|c|c|c|c|c|c|}\hline
	\multirow{2}{*}{\textbf{Gamma Dose}}& \multicolumn{8}{|c|}{\textbf{Sample Name}}\\ \cline{2-9}
	&50A &50B &50C &50D &80A &80B &80C &80D\\ \cline{2-9}
	\hline\hline
	0.1MGy& $\times$& {\scriptsize$\triangle$}& $\circ$& $\circ$& {\scriptsize$\triangle$}& {\scriptsize$\triangle$}& {\scriptsize$\triangle$}& $\circ$\\
	\hline
	1.0MGy& {\scriptsize$\triangle$}& $\circ$& $\circ$& $\circ$& $\times$& $\circ$& $\times$& $\circ$\\
	\hline
	2.5MGy& $\times$& $\circ$& $\circ$& $\circ$& $\circ$& $\circ$& $\circ$& $\circ$\\
	\hline
	\end{tabular}
	\label{tab:takasam}
\end{table}
%----------------------------------------------------------------------%
\subsection{暗電流とLED光応答測定}
$\gamma$線照射前後での暗電流とLED赤外光($\lambda\sim$850~nm)応答の変化を図\ref{fig:G}に示す.いずれのサンプルにおいても暗電流は1桁程度増加しているが,暗電流の増加量は増幅層の不純物濃度には依存しないことが分かった.これは,第\ref{sub:damage}節で前述した表面損傷によるものだと考えられる.いずれのサンプルにおいても,照射前後のLED光応答に大きな差は見られない.LGADの増幅機能は表面損傷では失われないことが確かめられた.
\begin{figure}[H]
\begin{minipage}{0.5\hsize}
	\centering
	\includegraphics[width=75mm]{./graph/G50B.pdf}
\end{minipage}
\begin{minipage}{0.5\hsize}
	\centering
	\includegraphics[width=75mm]{./graph/G80B.pdf}
\end{minipage}
\begin{minipage}{0.5\hsize}
	\centering
	\includegraphics[width=75mm]{./graph/G50C.pdf}
\end{minipage}
\begin{minipage}{0.5\hsize}
	\centering
	\includegraphics[width=75mm]{./graph/G80C.pdf}
\end{minipage}
\begin{minipage}{0.5\hsize}
	\centering
	\includegraphics[width=75mm]{./graph/G50D.pdf}
\end{minipage}
\begin{minipage}{0.5\hsize}
	\centering
	\includegraphics[width=75mm]{./graph/G80D.pdf}
\end{minipage}
 	\caption{$\gamma$線照射前後の暗電流変化.}
	\label{fig:G}
\end{figure}
%----------------------------------------------------------------------%
\newpage
\section{中性子線照射}
ピクセル型LGADのバルク損傷を評価するため,2016年12月にSloveniaのLjubljanaにあるヨージェフ・ステファン研究所にて,中性子線照射を行ったサンプルを用いて測定を行った.\par
\subsection{中性子線照射環境と照射量}
Ljubljanaへ発送するための準備を2016年12月8日に高エネルギー加速器研究機構で行い,翌9日に発送した.12月14日にLjubljanaに到着したサンプルは,翌15日に
%現地にいるIgor Mardic,Vladimir Cindro両名によって
中性子線照射された.Ljubljanaからは2017年1月3日に発送され,6日に高エネルギー加速器研究機構に到着するまでの3日間は0~℃に保たれていた.それ以降は,作業時と測定時を除いて$-$20~℃で管理した.\par
照射量と照射したサンプルの一覧を表\ref{tab:neusam}に示す.ただし,{\scriptsize$\triangle$}は実験の過程でスパークしたサンプルである.50Aのサンプルについては測定が終わっていないため空欄になっている.
\begin{table}[H]
	\centering
	\caption{ピクセル型センサーの中性子線照射サンプル一覧.}
	\vspace{5truemm}
	\begin{tabular}{|c||c|c|c|c|c|c|c|c|}\hline
	\multirow{2}{*}{\textbf{Neutron Fluence}}& \multicolumn{8}{|c|}{\textbf{Sample Name}}\\ \cline{2-9}
	&50A &50B &50C &50D &80A &80B &80C &80D\\ \cline{2-9}
	\hline\hline
	0.3$\times$10$^{15}$~n$_{eq}$/cm$^{2}$& & {\scriptsize$\triangle$}& {\scriptsize$\triangle$}& $\circ$& {\scriptsize$\triangle$}& {\scriptsize$\triangle$}& {\scriptsize$\triangle$}& $\circ$\\
	\hline
	1.0$\times$10$^{15}$~n$_{eq}$/cm$^{2}$& & $\circ$& $\circ$& $\circ$& {\scriptsize$\triangle$}& $\circ$& $\circ$& $\circ$\\
	\hline
	3.0$\times$10$^{15}$~n$_{eq}$/cm$^{2}$& & $\circ$& $\circ$& $\circ$& $\circ$& $\circ$& $\circ$& $\circ$\\
	\hline
	\end{tabular}
	\label{tab:neusam}
\end{table}
%----------------------------------------------------------------------%
\subsection{暗電流とLED光応答測定}
中性子線照射後のサンプルの電流を照射量で比較するためには,60~℃で80分アニーリングさせる必要がある\cite{moll}.そこで測定は60~℃で80分のアニーリング後に行なった.また,LGADの増幅機能が残っているかを判断しやすいように,第\ref{sub:temp}項で示した温度に関する依存性と第\ref{sub:wave}項で示した波長に関する依存性から,より低い温度($-$40~℃)で,より長い波長($\sim$850~nm)の赤外LEDを用いて測定を行なった.中性子線照射前後での暗電流とLED赤外光応答の変化を図\ref{fig:N}に示す(他のサンプルについては付録\ref{neuN}参照).第\ref{sub:damage}節で前述したように,照射量に伴い暗電流は増加するはずであるが,0.3$\times$10$^{15}$~n$_{eq}$/cm$^{2}$の照射量に関しては,明らかに他の照射量に比べて大きい.50C・50D・80Dいずれのサンプルにおいても同程度に大きくなっていることから,表面の汚れ等による増加とは考え難い.50C・50Dのサンプルでは500~V以降で1.0$\times$10$^{15}$~n$_{eq}$/cm$^{2}$が3.0$\times$10$^{15}$~n$_{eq}$/cm$^{2}$と同程度まで増加しているが,これは低い照射量の方がLGADの増幅機能が低電圧においても十分残っているために,暗電流の増幅が見えていると解釈できる.
LED光応答では,照射量が多いほど増幅機能が見える電圧値が高くなっているが,これは従来型のN$^{+}$-in-Pセンサーでも中性子線照射時のバルク比抵抗の低下に伴い見られる高電圧へのシフトである.特にActive Thicknessの薄い50C・50Dについては,未照射時の増幅が見られる電圧値が低いために,高照射後でもLGADの増幅機能が残っていることがIV特性からも分かる.一方,Active Thicknessの厚いサンプルでは,照射量1.0$\times$10$^{15}$~n$_{eq}$/cm$^{2}$を越えるとはっきりと増幅が見られるものはないが,高電圧において電流値の上昇は見ることができる.増幅率の変化を表\ref{tab:gain}にまとめた.ただし,$-$100~Vでの電流値を基準とした.\par
\begin{table}[H]
	\centering
	\caption{増幅率が10となる電圧値の変化.}
	\vspace{5truemm}
	\begin{tabular}{|c||c|c|c|c|}\hline
	\multirow{2}{*}{\textbf{Sample}}& \multicolumn{4}{|c|}{\textbf{Neutron Fluence}}\\ \cline{2-5}
	&Non Irradiated &0.3$\times$10$^{15}$~n$_{eq}$/cm$^{2}$ &1.0$\times$10$^{15}$~n$_{eq}$/cm$^{2}$ &3.0$\times$10$^{15}$~n$_{eq}$/cm$^{2}$\\ \cline{2-5}
	\hline\hline
	50C& 380V& 550V& 700V& $\textgreater$700V\\
	\hline
	50D& 260V& 470V& 640V& $\textgreater$700V\\
	\hline
%	80A& 610V& & & $\textgreater$800V\\
%	\hline
%	80B& 630V& & $\textgreater$800V& $\textgreater$800V\\
%	\hline
	80C& 540V& & $\textgreater$800V& $\textgreater$800V\\
	\hline
	80D& 350V& 760V& $\textgreater$800V& $\textgreater$800V\\
	\hline
	\end{tabular}
	\label{tab:gain}
\end{table}
未照射時の増幅が低電圧で見られる方が,放射線照射後も低電圧で増幅が見られることを確認した.つまり,増幅層の不純物濃度が高く,Active Thicknessの薄いほど,放射線耐性も高いと言える.\par
\begin{figure}[H]
\begin{minipage}{0.55\hsize}
	\centering
	\includegraphics[width=75mm]{./graph/NLE50C.pdf}
\end{minipage}
\begin{minipage}{0.45\hsize}
	\centering
	\includegraphics[width=75mm]{./graph/NLI50C.pdf}
\end{minipage}
\begin{minipage}{0.55\hsize}
	\centering
	\includegraphics[width=75mm]{./graph/NLE50D.pdf}
\end{minipage}
\begin{minipage}{0.45\hsize}
	\centering
	\includegraphics[width=75mm]{./graph/NLI50D.pdf}
\end{minipage}
\begin{minipage}{0.55\hsize}
	\centering
	\includegraphics[width=75mm]{./graph/NLE80C.pdf}
\end{minipage}
\begin{minipage}{0.45\hsize}
	\centering
	\includegraphics[width=75mm]{./graph/NLI80C.pdf}
\end{minipage}
\begin{minipage}{0.55\hsize}
	\centering
	\includegraphics[width=75mm]{./graph/NLE80D.pdf}
\end{minipage}
\begin{minipage}{0.45\hsize}
	\centering
	\includegraphics[width=75mm]{./graph/NLI80D.pdf}
\end{minipage}
 	\caption{中性子線照射前後のIV特性変化(暗電流:左,光応答:右).}
	\label{fig:N}
\end{figure} 	% 本文4
%%----------------------------------------------------------------------%
\chapter{結論}
%----------------------------------------------------------------------%
時間分解能とS/N比に優れたLGAD検出器の研究開発を行なっている.本研究では,浜松ホトニクス社が製造したLGADの構造理解を含めた基礎特性の評価を行なった.IV 特 性の波長依存性から増幅層の厚さが3-20~$\mu$m程度以上であると推定し,CV 特性から空乏化はN$^{+}$電極と接するバルク部の境界,N$^{+}$電極と増幅層との境界,バルク部全体の順で進むことを推察した.
ピクセル型センサーの放射線耐性をγ線照射と中性子線照射について,照射前後での IV 特性の変化から評価した.LGADの増幅機能は,2.5~MGyまでのγ線照射による表面損傷では,大きな影響を受けないことが分かった.3.0$\times10^{15}$~n$_{eq}$/cm$^{2}$までの中性子線照射によるバルク部損傷では,増幅機能は保持されているものの,未照射に比べて高電圧を加えなければ増幅が見られなくなった.
今後はより明確な増幅率の評価のために,レーザー・$\beta$線による電荷収集量測定を行なっていく必要がある.LGADの時間分解能の良さを評価する方法についても検討しなければならない.	% 本文5
%
%\appendix
%\chapter{}
\section{IV特性その他の図}
\subsection{温度依存性}
\label{T}
\begin{figure}[H]
\begin{minipage}{0.5\hsize}
	\centering
	\includegraphics[width=75mm]{./graph/Tdep50C.pdf}
\end{minipage}
\begin{minipage}{0.5\hsize}
	\centering
	\includegraphics[width=75mm]{./graph/Tdep50D.pdf}
\end{minipage}
 	\caption{IV特性の温度依存性(Active Thickness 50~$\mu$m).}
\end{figure}

\subsection{波長依存性}
\label{W}
\begin{figure}[H]
\begin{minipage}{0.55\hsize}
	\centering
	\includegraphics[width=75mm]{./graph/WLdep50Cfull.pdf}
\end{minipage}
\begin{minipage}{0.45\hsize}
	\centering
	\includegraphics[width=75mm]{./graph/WLdep50C100.pdf}
\end{minipage}
\begin{minipage}{0.55\hsize}
	\centering
	\includegraphics[width=75mm]{./graph/WLdep50Dfull.pdf}
\end{minipage}
\begin{minipage}{0.45\hsize}
	\centering
	\includegraphics[width=75mm]{./graph/WLdep50D100.pdf}
\end{minipage}
 	\caption{IV特性の波長依存性(Active Thickness 50~$\mu$m).}
\end{figure}

\section{CV特性その他の図}
\label{CV}
\begin{figure}[H]
\begin{minipage}{0.5\hsize}
	\centering
	\includegraphics[width=75mm]{./graph/60GB50CVfull.pdf}
\end{minipage}
\begin{minipage}{0.5\hsize}
	\centering
	\includegraphics[width=75mm]{./graph/60GB50CV100.pdf}
\end{minipage}
 	\caption{CV特性その他の図.}
\end{figure}

\section{中性子線照射後IV特性その他の図}
\label{neuN}
\begin{figure}[H]
\begin{minipage}{0.55\hsize}
	\centering
	\includegraphics[width=75mm]{./graph/NLE50B.pdf}
\end{minipage}
\begin{minipage}{0.45\hsize}
	\centering
	\includegraphics[width=75mm]{./graph/NLI50B.pdf}
\end{minipage}
\begin{minipage}{0.55\hsize}
	\centering
	\includegraphics[width=75mm]{./graph/NLE80A.pdf}
\end{minipage}
\begin{minipage}{0.45\hsize}
	\centering
	\includegraphics[width=75mm]{./graph/NLI80A.pdf}
\end{minipage}
\begin{minipage}{0.55\hsize}
	\centering
	\includegraphics[width=75mm]{./graph/NLE80B.pdf}
\end{minipage}
\begin{minipage}{0.45\hsize}
	\centering
	\includegraphics[width=75mm]{./graph/NLI80B.pdf}
\end{minipage}
 	\caption{中性子線照射前後の暗電流変化.}
	\label{fig:neuN}
\end{figure}		% 付録
%
%\begin{acknowledgment}
本卒業論文は,筆者が筑波大学 理工学群 物理学類在学中に素粒子実験研究室において行なった研究をまとめたものです.本研究を進めるにあたり,多くの方々からお力添えをいただきました.ここに深謝の意を表します。\par
指導教員である受川史彦先生には,論文の書き方や発表の仕方をはじめとし,多くのご助言をいただきました.最後の最後まで辛抱強くご指導いただき,有難うございます.実験の担当教員である原和彦先生には,研究への心構えから実際の測定方法まで,本当に多くのことを教えていただきました.心が折れそうなときにも測定を続けてこられたのは,先生のご支援と励ましのお言葉があったからこそだと思っております.心より感謝いたします.佐藤構二先生には,第\ref{sub:Laser}節に述べた測定について,多くの示唆をいただきました.今後ともよろしくお願いいたします.\par
また,金信弘先生や武内勇司先生,大川英希先生には,日頃より素粒子実験に関するお話を聞かせていただき,知見を広げることができました.有難うございました.\par
花垣和則先生はじめATLAS日本シリコングループの皆様には,ミーティングにおける私の拙い報告にも耳を傾け,ご助言いただきましたことを感謝いたします.特に海野義信先生には,本研究の方針や測定結果の理解についてご指導いただきました.初歩的な疑問にもひとつひとつ答えていただき,有難うございました.今後とも,ATLAS日本シリコングループに少しでも貢献できるよう頑張っていきます.\par
研究員の武政健一さんには,研究室での生活において右も左もわからないところから,研究室でのいろはを教えていただきました.秘書の服部千尋さんには,出張時の手続き等を手伝っていただいたり,研究室での生活が円滑に進むよう支えていただきました.研究室に同期がいない中,多くの先輩方にも声をかけていただき,気にかけていただいたこと,大変感謝しております.\par
最後に,自分の思う道に進む機会を与え,全力で学業に励めるよう支援してくれた両親に対し,深い感謝の意を表して謝辞といたします.
\begin{flushright}
和田 冴
\end{flushright}
\end{acknowledgment}	% 謝辞。要独自コマンド、include先参照のこと
%\begin{thebibliography}{30}
%
\bibitem{cern}CERN,
\newblock http://home.cern
%
\bibitem{atlas}ATLAS Experiment,
\newblock http://atlas.cern/discover/detector
%
\bibitem{D}Martin A. Green.
\newblock \textit{Self-consistent optical parameters of intrinsic silicon at 300 K including temperature coefficients}.
\newblock Solar Energy Materials $\&$ Solar Cells 92 (2008) 1305-1310.
%
\bibitem{moll}Michael Moll.
\newblock \textit{Radiation Damage in Silicon Particle Detectors − microscopic defects and macroscopic properties −}.
\newblock PhD thesis, University of Hamburg, 1999.
%
\bibitem{neutron}Luka Snoj, Gasper Zerovnik, Andrej Trkov.
\newblock \textit{Computational analysis of irradiation facilities at the JSI TRIGA reactor}.
\newblock Jozef Stefan Institute, Jamova cesta 39, SI-1000 Ljubljana, Slovenia.
\newblock 18 November 2011.
%
\bibitem{leo}William R. Leo.
\newblock \textit{Techniques for Nuclear and Particle Physics Experiments}.
\newblock Springer- Verlag Berlin Heidelberg GmbH.
%
\bibitem{gamma}高崎量子応用研究所.
\newblock $http://www.taka.qst.go.jp/tiara/665/index_Co.php$
%
\bibitem{silicon}大山英典, 葉山清輝.
\newblock 半導体デバイス工学
\newblock 森北出版株式会社, 2004.
%
\end{thebibliography}	% 参考文献。要独自コマンド、include先参照のこと

\end{document}
