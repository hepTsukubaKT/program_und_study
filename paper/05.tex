%----------------------------------------------------------------------%
\chapter{結論}
%----------------------------------------------------------------------%
時間分解能とS/N比に優れたLGAD検出器の研究開発を行なっている.本研究では,浜松ホトニクス社が製造したLGADの構造理解を含めた基礎特性の評価を行なった.IV 特 性の波長依存性から増幅層の厚さが3-20~$\mu$m程度以上であると推定し,CV 特性から空乏化はN$^{+}$電極と接するバルク部の境界,N$^{+}$電極と増幅層との境界,バルク部全体の順で進むことを推察した.
ピクセル型センサーの放射線耐性をγ線照射と中性子線照射について,照射前後での IV 特性の変化から評価した.LGADの増幅機能は,2.5~MGyまでのγ線照射による表面損傷では,大きな影響を受けないことが分かった.3.0$\times10^{15}$~n$_{eq}$/cm$^{2}$までの中性子線照射によるバルク部損傷では,増幅機能は保持されているものの,未照射に比べて高電圧を加えなければ増幅が見られなくなった.
今後はより明確な増幅率の評価のために,レーザー・$\beta$線による電荷収集量測定を行なっていく必要がある.LGADの時間分解能の良さを評価する方法についても検討しなければならない.